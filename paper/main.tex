\documentclass[zihao=5]{ctexart}
% 导言区 主要进行一些全局设置
\usepackage{ctex}
\usepackage{amsmath,amsfonts,amssymb}
\usepackage{geometry}                                               
\geometry{a4paper,scale=0.8}
\usepackage{booktabs}
\usepackage{caption}
\usepackage{subcaption}
\usepackage{graphicx}
\usepackage{float}
\usepackage{bm}
\usepackage{multirow}
\usepackage{multirow}
\usepackage{tabularx}

\ctexset{
   section={
   name={第,章},
   number=\chinese{section},
   format={\Large\centering\heiti}
   },
   subsection={
   name={\bfseries\S},
   format={\large\centering\heiti}
   },
   subsubsection={
   name={\bfseries},
   format={\heiti}
   }
}
\usepackage{fancyhdr}

\bibliographystyle{plain}
% \bibliographystyle{unsrt}

\pagestyle{fancy}
\fancyhead[R]{\rightmark} % 在偶数页的右侧显示章名
\fancyhead[LE,RO]{} 

\renewcommand\sectionmark[1]{
\numberwithin{figure}{section}
\numberwithin{equation}{section}
\renewcommand{\thesubfigure}{\alph{subfigure}}

\markright{\CTEXifname{\CTEXthesection——}{}#1}}
\addtolength{\voffset}{0.5cm} \addtolength{\textheight}{-0.5cm}
\addtolength{\hoffset}{0.4cm}

\title{南京审计大学硕士研究生毕业论文 \\—— $L_1$范数主成分分析及其在宏观经济因子模型中的应用}
\author{蒯强}
\date{\today}

\begin{document}
   \maketitle 
   \newpage
   \tableofcontents
   \thispagestyle{empty} 
   \newpage
   \setcounter{page}{1}

   \nocite{*}
   \section{绪论}
本章阐述了本篇论文的研究背景和研究意义,综述了国内外学者在相关领域的研究现状、主要的成果、结论和观点。
由此展开,叙述了论文的主要研究内容、研究所使用的技术路线、本文的主要创新点和行文的结构。

\subsection{研究背景及意义}
在宏观经济学理论中,政府是指导和调控经济运行的主体。国民经济的发展和政府的指导和调控紧密相关,政府部门需要时刻把握国民经济的方方面面,
不仅仅要关心主要产业GDP的增长,还要关心经济中的通货膨胀、利率和汇率等等许多复杂的经济变量,并且需要研究这些经济变量发生变化的原因
以及各种经济变量之间的作用关系。只有这样,政府才能发现经济中存在的问题,并且给出针对性的指导和调控手段。

在古典经济学理论中,对于国民经济作为一个整体的活动情况一般不予考察,古典经济学往往从围观主体出发来解释观察到的宏观经济问题。然而,在市场经济
高度发展的当代,市场规模空前扩大,对人类社会已经产生了深远影响。宏观经济的发展已经密切关系到民生发展,宏观经济的动荡也通过金融危机和失业
直接关系到社会稳定。特别是一战后的大萧条使得古典经济学对宏观经济的分析陷入了困境,在凯恩斯理论的影响下,现代宏观经济学进入主流。
在现代宏观经济学的主张下,政府应该采取随机应变的宏观经济政策,通过财政和货币手段对国民经济发展进行调控和指导。

政府管理国民经济的关键是制定正确的经济政策,而做决策的前提就是预测。因此制定经济政策不仅需要各级政府对经济发展的当前状态有充分把握,
并且能够对经济的未来变化有一定的预测能力。经济预测是根据经济发展的时间和空间纬度的综合信息来试图研究经济变化趋势的技术。政府为了提高
经济政策的有效性,就需要通过更加准确的预测来分析和判断信息,因此政府部门需要不断追求更加准确的经济预测技术,才能在复杂的经济形式变化中
掌握主动权。

除了经济发展的需要,为了规避国民经济的风险,把控我国宏观经济安全大局,也对政府更加准确地把握和预测国民经济提出了要求。
当前我国正处于经济和社会向前发展的关键时期,经济虽然多年持续高速发展,但是,经济结构发展不均衡的现象逐渐显现,
社会经济系统的复杂性也日渐增强。这就更需要对经济系统进行实时准确的检测和对未来经济形势准确预测,提前做好准备工作,
防范经济增长的重大风险。经济风险不仅仅和宏观经济运行状况紧密关联,同样和国计民生密不可分,
不管是通过对经济系统的宏观把控还是对经济运行中某些指标的局部观察,我们都可以了解经济运行的一些趋势,
并且通过对经济中的某些重要运行趋势进行预测,从而规避经济风险,稳定宏观经济安全大局。
把控宏观经济安全大局对于国民经济平稳快速发展、改善国计民生、促进经济又好又快发展、促进国家安全稳定有重要的战略和实际意义。
习近平总书记于2014年首次提出总体国家安全观,并构建了集政治安全、国土安全、军事安全、经济安全和文化安全等领域为一体的国家安全体系。
其中,经济安全是国家安全体系的重要组成部分,是国家安全的基础。宏观经济安全问题需研究1个或多个指标(如经济增长量)与其他指标(如资本、劳动力、人口和技术等)相互间的关系,
从而揭示经济现象背后的经济规律,发现经济运行风险点,制定经济政策。为了避免宏观经济中的冲击带来的风险,
需要对宏观经济总量进行实时预测和短期预测来减少不确定性,为宏观经济调控政策提供决策支持。通过对宏观经济指标进行有效监测和预测,
并提前识别和防范相关风险,对保证国民经济稳步增长具有重要意义。

宏观经济预测方法繁多,有的经济学家使用经济学方程组来给宏观经济系统建立模型,
从而通过模型来对某些宏观经济指标进行预测并且给出经济学分析。
然而国民经济是一个非常复杂的整体,它是一个内部极其复杂的系统,并且经常伴随难以预测的外部冲击,对于复杂的经济变量,想到做到准确预测往往非常艰难。
虽然现在传统的经济学理论和现代计算机技术的结合下,已经发展出了许多预测方法,并且学者们仍在不断利用新的理论和技术来完善这些已有的方法。然而,
在我国国情下,进行准确的经济预测仍然非常困难。现代经济系统极为复杂,
意味着使用这种方法我们就可能需要建立非常复杂的经济分析模型,有可能导致分析困难,并且往往预测结果和真实的经济运行结果有较大出入。
因为系统过于复杂,学者们就开始使用基于历史数据进行向前预测的经验方法,这种方法大量应用数据和数据分析、大数据技术和统计学知识,
并且融合了成熟的经济学理论和模型,往往能够起到不错的效果。在现代经济系统中,经济社会中每时每刻都在产生大量的数据,
这些海量的经济活动数据给经济预测提供了丰富的数据支撑,如何有效地利用大数据时代丰富的数据资源进行宏观经济现时预测已成为一个重要的研究课题。
当今的宏观经济数据具有高维度、数量多、处理困难等特征。结合计算机科学和统计学等方面的技术和理论知识,
我们能够采取一些方法来操纵数据,丛中过挖掘提炼出有效的信息。关于这个方面的研究正在火热进行,人们不断尝试使用更新的技术来处理数据,
提取信息,希望利用高维度下丰富多面的经济信息来对未来的经济发展趋势作较为准确的分析。

由于宏观经济变量往往具有维数很高、分布重尾、数据集不完整等特征,导致在宏观经济的数量研究中遇到稳健性问题。
例如线性回归模型的估计、主成分分析、因子模型的估计等场景下,基于最小二乘法的经典方法往往难以得到满意的结果。
因此,在宏观经济数据的分析和预测中往往还需要引入更加稳健的方法。使用更加稳健的的基于$L_1$范数的方法来代替
基于$L_1$范数的方法例如使用最小绝对值回归代替最小二乘法,是一个很常见的做法。

基于$L_1$范数的稳健方法在宏观经济预测中已经得到了一定的应用,例如最小绝对值回归等。
高维宏观经济数据可以通过因子模型来分析和处理,得到一些对于经济预测很有价值的信息,
然而由于经济收到外部冲击等原因,许多宏观经济指标不能够服从正态分布,往往在分布上呈现重尾特征。
这就影响了采用普通主成分分析方法估计因子模型,然而
然而在该领域相关问题还有待进一步研究。

\subsection{基本概念和研究现状综述}
本文的研究内容是$L_1$范数主成分分析及其在宏观经济因子模型中的应用。
因此主要围绕近似因子模型和$L_1$范数主成分分析两大主题进行了文献的搜集和整理,
下面简要介绍它们的基本概念,并梳理了关于这两个主题国内外学者在相关领域的研究现状和存在问题。

\subsubsection{因子模型及其在宏观经济中的应用}
因子模型最早由英国心理学家Charles Spearman提出,最早用于智力测验得分数据的分析中。
提出因子模型是为了使用少量潜在的、不可观测的因子来描述多个相依变量间的关系。
假设一组变量能够用因子模型来描述,那么我们就把这组变量称为观测变量,因为它们的值可以实际测度,
而观测变量的值是由一组不可观测的因子的线性组合决定的。

经典因子模型提出后,迅速在心理学、社会学等多个学科内得到了广泛的使用。
然而因子模型在宏观经济研究领域的应用出现较晚,这可能是因为经典因子模型主要用于处理截面数据,
而宏观经济研究中主要研究的数据为时间序列数据。
然而近年来因子模型的理论和应用已经得到了很大的完善和发展,
作为一种重要分析工具因子模型已经成为计量经济学的必修内容,并且动态因子模型
常被应用于宏观经济政策的评估、经济指数的构建和经济指标的预测中。

因子模型在宏观经济研究中的广泛使用和当今宏观经济数据集的特征密切相关。
从政府部门和一些经济研究机构公布的数据和报告可以发现,在制定宏观经济政策时,政策制定者
在决策过程中考虑到了非常多的经济变量。一个简单的经济政策背后可能需要分析成百上千不同的经济指标,
并且这些经济指标的可靠数据记录往往较少,许多宏观经济指标往往只有30年以内的可靠记录。
因此宏观经济研究中面对的数据较为特殊,常用的计量经济模型例如向量自回归模型等在这种数据集下
很难可靠地估计模型系数。在建立模型时将所有的经济指标都进行研究,大大提升了问题的计算复杂度,
难以快速给出分析结果。而因子模型可从高维数据集中提炼出少量公共因子,
这些因子中包含了经济运行的许多有效信息,因而可以通过分析这些公共因子准确
了解宏观经济的运行状况。

经典因子模型适用于截面数据,对于分析经济时间序列而言不十分适用,因此国内外学者们在经典因子模型上发展出了动态因子模型。
Geweke\cite{geweke1977dynamic}与Sargent\cite{sargent1977business}等人于1977年分别提出了动态因子模型,
动态因子模型是经典因子模型在时间序列数据上的延伸,它提供了从维数众多的经济时间序列数据中提取公共因子来研究和解释
经济波动的手段。
一般认为动态因子模型提出基于以下观点:宏观经济具有周期性的波动是通过一系列经济变量的活动来传递和扩散的,
任何一个经济变量本身的波动都不足以代表宏观经济的整体波动。对于经济波动的研究需要从多个具有相依性的经济变量同时着手。
因此,需要从一国许多经济时间序列数据中估计出驱动各变量波动的共同动态因子,并且给出解释。
动态因子模型已经成为判别和分析经济周期波动的有效工具。

动态因子模型主要应用于仅包含几个总体宏观经济变量的小型数据集,这主要是由于经济学家们假定影响所有变量的普遍性结构
冲击是经济变量协同变动的原因。但是在包含了成百上千个变量的高维宏观经济数据集中,常常存在某个仅影响某一组变量的部门冲击或是局部冲击,
若根据经济理 论将这些非普遍性冲击归于异质性部分,那么就可能造成异质性部分的截面相关,这又将违背经典因子模型中对于异质性部分相互正交的假定。
Chamberlain和Rothschild于1983年提出应该放松关于异质性部分协方差矩阵为对角阵的假定,即允许其为非对角阵,
从而发展出了近似动态因子模型\cite{chamberlain1982arbitrage}。在众多因子模型及其拓展模型中,近似动态因子模型得到了
最为广泛的使用。

Forni等人于2000年提出应该在近似动态因子模型的基础上,进一步放松假设,即允许观测变量收到因子的无限之后项的影响,从而
发展出了广义动态因子模型\cite{forni2000generalized}。但是在模型的参数估计方面,广义动态因子模型只能使用频域方法估计
因此其使用范围比较有限。

因子模型在宏观经济分析中有两个最主要的应用:1)构建指数。最早的尝试是Stock和Waston等于1989年提出的一致和先行景气指数
\cite{stock1989new}。另一个例子是基于Stock和Wasto在1999年提出的方法\cite{chan1999dynamic},构建了著名的CFNAI指数,它就基于美国的85个月度宏观经济数据
。
2)预测。Stock和Waston于2002年提出了扩散指数模型\cite{stock2002macroeconomic},使用因子的主成分估计量进行预测。
通过少量的因子,不仅利用了大量的信息而且避免了变量过多带来的的模型自由度问题。Eickmeier等
于2008年进行的实证研究\cite{eickmeier2008successful},使用美国宏观经济数据证明了使用扩散指数模型的预测效果要好于大部分其他方法。

目前在国外已经存在大量关于使用因子模型的进行宏观经济预测的实证研究(Luciani,2014\cite{luciani2014large}),而目前在国内使用因子模型进行宏观预测的实证研究
较少(高华川和张晓峒,2015\cite{高华川2015动态因子模型及其应用研究综述})。

\subsubsection{$L_1$范数主成分分析}
主成分分析是一种应用广泛的数据降维技术。通过选择高维数据在特征空间中的数个基向量构成投影,
这些基向量代表了高维数据在特征空间的投影方差最大的几个方向。对应到优化理论,传统主成分分析算法解决
$L_2$范数优化问题,因而也称为$L_2$范数主成分分析。

而面对噪声干扰高的数据,$L_2$范数主成分分析
不具有稳健性。而在优化理论中,$L_1$范数的优化问题对噪声不敏感,具有很好的稳健性。
若将$L_1$范数应用于主成分分析中,这种方法就称为$L_1$主成分分析,在理论上它具有很好的稳健性。
因此,国内外学者从很早就开始了对$L_1$范数主成分分析的研究。
众多应用性的研究(Galpin and Hawkins, 1987\cite{galpin1987methods};
Baccini et al.,1996\cite{baccini1996l1};Ke and Kanade, 2003\cite{ke2003robust};
Ding et al., 2006\cite{ding2006r};Choulakian, 2006\cite{choulakian2006l1};Gao, 2008\cite{gao2008robust}
;Kwak, 2008\cite{kwak2008principal})
已经表明,$L_1$范数主成分分析具有很强的抗噪声能力,应用于机器视觉\cite{ke2005robust}、
人脸识别\cite{kwak2008principal}等领域有非常好的效果。

然而$L_1$范数主成分分析在求解上比$L_2$范数主成分分析困难许多,
具体体现在难以获得全局最优解、计算复杂度高等方面。
$L_1$主成分分析有两种问题形式,一种是最小化重构误差的$L_1$范数;另一种是最大化特征空间投影的$L_1$的范数。

对于前者,
一个具有代表性的早期研究\cite{baccini1996l1}提出了一种基于典型相关分析的算法,该算法具有一定的启发意义;
Ke和Kanade等人于2005年提出了一种方法\cite{ke2005robust},该算法将全局的非凸优化问题使用
一种交替的凸优化算法解决,每次交替求解的问题可以获得全局最优,每一次交替中需要求解若干个线性规划问题;
Gao于2008年提出一种在Laplace噪声下估计最优$L_1$特征空间的Bayesian方法\cite{gao2008robust};
而Brooks等人于2013年提出一种算法
\cite{brooks2013pure},该算法逐次降低优化问题的维数,递进地寻找从$m$特征维空间向$m-1$特征维空间的最佳投射,而获得这样的最佳投射
需要求解一系列最小绝对值回归问题;Park等人于2016年提出一种算法\cite{park2016iteratively},在迭代中使用一种再加权最小二乘法求问题的近似解。
而对于该问题,至今没有一种算法能够给出全局最优解。

对于后者,Choulakian于2006年对该问题进行了完整描述\cite{choulakian2006l1},基于协方差矩阵的$L_1$估计\cite{galpin1987methods}来求解;
比较具有代表性的是Kwak于2008年提出的一种贪婪算法\cite{kwak2008principal},该算法在给出前$k$个主成分的局部最优估计下,
求解第$k+1$个主成分的最优解;Nie等人于2011年扩展了该算法,并给出一种采用非贪婪策略的实现方式\cite{nie2011robust};
Markopoulos等人于2014年给出了一种在特征维数已知情况下可获得全局最优解的算法\cite{markopoulos2014optimal},并且
证明了在原始数据维数和样本数一起趋于无穷时,本问题是一个NP难题,另外该结论同样被Mccoy\cite{mccoy2010two}等人证明;
Park等人于2020年提出了一种基于聚类——迭代拆解的解法\cite{park2021optimization},在没有进行理论论证的情况下进行了数值模拟,
实验结果表明了该算法在性能上有所改善。

\subsection{研究创新}

1)近似因子模型在宏观经济研究中被广泛使用,其因子的主成分估计量在宏观经济预测中能够起到很好的效果。
然而,传统主成分分析本质上求解是$L_2$范数优化问题,面对高维重尾宏观经济变量在数据处理上不具有稳健性。
$L_1$范数主成分分析作为$L_2$主成分分析的一种稳健替代手段,在信号处理、机器视觉和人脸识别中
已经有了大量应用。
本文创新性地采用$L_1$范数主成分分析作为近似因子模型的估计,来代替$L_2$主成分分析。
采用$L_1$范数主成分分析估计得到的因子和$L_2$是不同的,本文通过基于国内高维月度宏观经济数据集的实证研究,
比较了两种不同因子在宏观经济预测中的表现,论证了将$L_1$主成分分析作为近似因子模型估计来进行宏观
经济预测的可行性。

2)并且$L_1$范数在宏观经济预测中的一大应用是采用最小绝对值回归建立线性模型,而该模型
求解采用的线性规划方法在当今数据维数越来越高、数据量越来越大的场合下显得性能不足,因而限制了
最小绝对值回归在这一领域的应用。本文阐述了两种适用于最小绝对值回归性能优化的最新算法,
它们在应对大规模和高维数据集上较传统线性规划方法有着更好的性能表现,
本文通过数值模拟实验测试和分析了两种算法的优缺点,并且给出其在实际应用中的一些建议。
并在两种算法的基础上,融合彼此的优点提出了一种新的算法,通过模拟实验说明,该算法
在一定程度上同样可以提升最小绝对值回归的计算性能。

3)本文对$L_1$范数主成分分析的算法根据目标问题不同分,可以有两种不同的解法。
本文分别介绍了一种求解重构误差最小化的交替凸优化算法和一种求解特征空间$L_1$范数最大化的算法,
在国内月度宏观经济数据集上使用两种算法分别进行了近似因子估计,结果表明两者同样适用,并给出一些应用建议。
最后,本文结合对最小绝对值回归性能的研究,改进了交替凸优化算法,使其在估计高维近似因子模型时拥有更好的性能。

\subsection{本文结构}

本文主要内容分为5个章节:

第一章为绪论。该章节介绍了本文的写作背景和课题的研究意义,并且分析了课题相关的理论和应用研究
的进展和不足,指出了本文写作的创新点。

第二章为基于$L_1$范数主成分分析的高维因子模型与估计方法。该章节介绍了因子模型的基本概念,
提出了将$L_1$范数主成分分析引入近似因子模型的估计,介绍了一种求解$L_1$范数主成分分析
的交替凸优化算法并给出了具体步骤。
接下来,通过模拟实验探究$L_1$范数主成分分析相比于
传统基于$L_2$范数的主成分分析的稳健性。
最后,通过一个基于国内月度高维宏观经济数据
的实证研究,来验证$L_1$主成分分析能否应用在宏观经济预测中。

第三章为最小绝对值回归的性能研究。
该章节介绍了最小绝对值回归,它是稳健$L_1$范数在宏观经济建模中的另一个主要应用。
我们说明了最小绝对值回归的稳健性,介绍了其基本求解方法,并针对其求解性能问题进行了研究。
这里介绍了两种适用于高维最小绝对值回归的最新算法:聚类——迭代拆解算法和
基于替代变量的估计方法,详细阐述了两种算法的基本思想和具体步骤。
通过模拟高维宏观经济数据的数值实验,测试了算法的优缺点,并给出建议。
本章的研究有助于我们改进$L_1$主成分分析的交替凸优化算法的求解性能。

第四章为$L_1$范数主成分分析求解算法。
第二章已经将$L_1$范数主成分分析引入了近似因子模型的估计,并探究其应用于经济预测中的可行性。
这一章将进一步讨论$L_1$范数主成分分析,较深入地研究其求解问题。
这里首先利用第三章的研究成果
对交替凸优化算法的求解性能做出了优化。我们进行了数值模拟实验,
来测试算法的表现。
并且我们介绍了一种解决低维投影$L_1$范数最大化问题的求解算法,它和前面的交替凸优化算法
求解$L_1$主成分分析的角度有所不同,
我们同样进行了基于国内月度宏观经济数据的实证研究,探究该方法是否同样适用于宏观经济预测中,
并对实际应用中$L_1$算法的选择提出了建议。

第五章为结论与展望。
该章节总结了本文的主要研究结论和存在的问题,并为进一步的研究提出了方向。

   \section{基于$L_1$范数的因子模型估计}

\subsection{因子分析简介}
因子分析是主成分分析的推广和发展,它也是多元统计中广泛使用的一种降维方法。
因子分析研究相关矩阵或者协方差矩阵的内部依赖关系,它将多个变量综合为少数几个因子,
以再现原始变量和因子之间的关系。

因子分析最早由英国心理学家Charles Spearman提出,最早用于智力测验得分数据的分析中。
现在因子分析在心理学、社会学、经济学等学科都取得了广泛应用。
\subsubsection{正交因子模型}

设$\bm{X} = (X_1, X_2, ..., X_p)^T$是可观测的随机向量,$\mathbb{E}\bm{X} = \bm{\mu}$,
$\mathbb{D}(\bm{X}) = \bm{\Sigma}$,设$\bm{f} = (f_1, f_2, ..., f_m)^T\\ (m <p)$为不可观测的随机向量,且
$\mathbb{E}\bm{f} = 0$,$\mathbb{D}(\bm{f}) = \bm{I}_m$(即$\bm{f}$的各分量方差为1,且互不相关)。最后,设
$\bm{e} = (e_1, e_2, ..., e_p)^T$和$\bm{f}$互不相关,并且
$$
    \mathbb{E}(\bm{e}) = \bm{0}\mbox{,}\mathbb{D}(\bm{e}) = diag({\sigma _1}^2, ..., {\sigma _p}^2)
    \mbox{为对角矩阵。}
$$
假设随机向量$\bm{X}$满足以下模型:
\begin{equation*}
\left\{
\begin{array}{clr}
    X_1 - \mu_1 = a_{11}f_1 + a_{12}f_2 + ... + a_{1m}f_m + e_1, \\
    X_2 - \mu_2 = a_{21}f_1 + a_{22}f_2 + ... + a_{2m}f_m + e_2, \\
    ... \\
    X_p - \mu_p = a_{p1}f_1 + a_{p2}f_2 + ... + a_{pm}f_m + e_p,
\end{array}
\right.
\end{equation*}
我们称该模型为正交因子模型,矩阵表示为:
$$
    \bm{X} = \bm{\mu} + \bm{A}\bm{f} + \bm{e}
    \eqno{(2-1)}
$$
其中$f_1, f_2, ..., f_m$称为$\bm{X}$的公共因子;$e_1, e_2, ..., e_p$称为$\bm{X}$
的特殊因子。公共因子一般对$\bm{X}$的每一个分量都起作用,而$e_i$一般仅仅对$X_i$起作用。
并且各个特殊因子之间以及特殊因子和所有公共因子之间都是不相关的。
其中$\bm{A} = (a_{ij})_{p \times m}$是待估计的系数矩阵,称为因子载荷矩阵。
$a_{ij}$称为第$i$个变量在第$j$个因子上的载荷,它反映了第$i$个变量在第$j$个因子上的相对重要性。

% 设$p$维随机向量$X=(X_1, X_2, ..., X_p)^T$的数学期望为$\mu = (\mu_1, \mu_2, ..., \mu_p)^T$,
% 协方差矩阵为$\Sigma$,假设$X$线性依赖于少数几个不可观测的随机变量$f_1, f_2, ..., f_m(m < p)$,和$p$个随机
% 误差项$e_1, e_2, ..., e_p$,一般称$f_1, f_2, ..., f_m$为公共因子,称$e_1, e_2, ..., e_p$为特殊因子或
% 误差,因子模型有如下表达式:
% $$ X = \mu + Af + e \label{a}  \eqno{(2-1)}$$
% 其中$f = (f_1, f_2, ..., f_p)^T$为因子,$A$是因子载荷矩阵,$e = (e_1, e_2, ..., e_p)^T$是特殊因子。
% 在式($2-1$)中,随机向量$X$围绕均值的波动由公共因子的线性组合加上一个特殊因子解释。
% 经典因子模型假设相互独立,$e_1, e_2, ..., e_p$相互独立并且$f$和$e$的样本之间相互独立。

\subsubsection{动态因子模型}

经典因子模型即正交因子模型提出后,迅速成为在心理学、社会学等多个学科内得到了广泛的使用。
然而因子模型在计量经济学研究领域的应用出现较晚,这可能是因为经典因子模型主要用于处理截面数据,
而计量经济学,尤其是宏观计量经济学的主要研究对象为时间序列数据。

Geweke与Sargent等人首先在计量经济学领域提出了动态因子模型,
动态因子模型是经典因子模型在时间序数据上的延伸,它提供了从维数众多的经济时间序列数据中提取公共因子来研究和解释
经济波动的手段。
动态因子模型的基本思想是:宏观经济具有周期性的波动是通过一系列经济变量的活动来传递和扩散的,
任何一个经济变量本身的波动都不足以代表宏观经济的整体波动。对于经济波动的研究需要从多个具有相依性的经济变量同时着手。
因此,需要从一国许多经济时间序列数据中估计出驱动各变量波动的共同动态因子,并且给出解释。
动态因子模型已经成为判别和分析经济周期波动的有效工具。

令$\bm{X}_t = (\bm{X}_{1t},\bm{X}_{2t}, ..., \bm{X}_{pt})^T$为一组宏观经济变量在$t$时刻的水平,并且$\bm{X}_t$可以表达为如下形式:
$$ \bm{X}_t = \lambda(L)\bm{f_t} + e_t \eqno{(2-2)}$$
其中,$\bm{f}_t$ 为$q\time 1$ 维的动态因子向量,$\lambda(L)$为由$s$阶滞后多项式算子组成的$p \times q$矩阵。
动态因子模型通常假设动态因子向量服从某一个向量随机过程。即动态因子模型
不仅允许观测变量受因子滞后项的影响,而且也允许因子本身具有独立的动态演化过程。

\subsubsection{近似因子模型}
在近似因子模型出现以前,DFM 主要应用于仅 包含几个总体宏观经济变量的小型数据集。这主要 是由于经济学家们假定影响所有变量的普遍性结构
冲击是经济变量协同变动的原因。然而在包含成百 上千个变量的大型经济数据集中,往往存在仅影响 某一组变量的部门冲击或局部冲击,
若根据经济理 论将这些非普遍性冲击归于异质性部分,则会造成 异质性部分截面相关,这又违背了经典因子模型中 异质性部分相互正交的假定。

因此,Chamberlain 和 Rothschild放弃 异质性部分的协方差矩阵为对角阵 的假定,允许异质性部分存在一定程度的截面相关, 
模型假设允许$f_t$的各个分量$f_1, f_2, ..., f_m$具有相依性,$e_1, e_2, ..., e_p$可以不独立,
并且允许$f_t$,$e_t$具有时间序列相依性。从而将模型扩展为了为动态近似因子模型。

% 通过放宽模型假设和形式变换,可以将动态因子模型转变为在经济变量预测中更加实用的静态近似因子模型。
% 令$X_t = (X_{1t},X_{2t}, ..., X_{pt})^T$为一组宏观经济变量在$t$时刻的水平,并且$X_t$可以表达为如下形式:
% 其中$f_t$是$q$维的动态相依因子向量,$\lambda(L)$为由$s$阶滞后多项式算子组成的$p \times q$矩阵。近似因子
% 模型的模型假设允许$f_t$的各个分量$f_1, f_2, ..., f_m$具有相依性,$e_1, e_2, ..., e_p$可以不独立,


% 令某一个经济指标$y$在$t+h$时刻的水平为$y_{t+h}$,则其预测值由式($2-4$)给出:
%     $$X_t = AF_t + e_t $$
%     $$y_{t+h} = \beta F_t + \beta _yy_t + \varepsilon _{t+h} \eqno{(2-4)}$$
% 其中$\beta F_t$包括了近似因子$t$时刻的当期项和滞后项,$\beta _yy_t$代表了指标$y$受自身滞后项的影响,$\varepsilon _{t+h}$
% 为预测误差。

\subsection{因子模型的估计}
% 在$2.1$节中提出,使用因子模型进行分析的关键是找出最具贡献的因子,而关键步骤就是要估计因子载荷矩阵。
% 对于($2-1$)中的因子模型,假设我们已知$p$个相关变量的观测值
% $$\bm{X}_{(i)} = (x_{i1}, ..., x{i_p})^T (i = 1, 2, ..., n)$$
对于正交因子模型,
在因子载荷矩阵$\bm{A}$中,我们计算各列的平方和,记为$q_j^2$,即
$$
    q_j^2 = \sum_{i=1}^p a_{ij}^2 (j = 1, 2, ..., m)
$$
$q_j^2$表示第$j$个公共因子$f_j$对$\bm{X}$的所有分量的总影响,称为公共因子$f_j$对$\bm{X}$的
方差贡献。因此如果我们列出$\bm{A}$的所有列平方和,按照方差贡献大小就可以选出最有影响的公共因子。
因此因子分析的关键步骤就是估计出因子载荷矩阵。

对于正交因子模型可以通过主成分法、主因子法和极大似然法来估计因子载荷矩阵。不可观测的公共因子有时候也需要进行估计,
例如用来诊断因子模型或者作为进一步分析的原始数据,这就需要我们给出因子得分,一般在得出了因子载荷矩阵之后,
对于因子得分,可以通过加权最小二乘法(巴特莱因子得分)或者通过回归法(汤普森因子得分)进行估计。

但是对于动态因子模型和近似动态因子模型,因为不满足正交因子模型的一些关键模型假设,因此需要不同的估计手段。

\subsubsection{动态因子模型的估计}
动态因子模型的提出者Geweke和Sargent等人都是使用频域方法估计模型,这种方法不能够直接估计出动态因子$\bm{f}_t$,
因此也就不能将因子作为预测或者扩展模型等用途。使得动态因子模型的应用收到限制。
由于动态近似因子模型估计困难,因此在预测中,往往使用模型的静态形式:
$$X_t = AF_t + e_t \eqno{(2-3)}$$
式($2-3$)中,$F_t$是$m$维向量,称为静态因子,即$F_t$仅在当期影响$X_t$(因为$F_t$包括了动态因子$f_t$的当期项和
滞后项),它本身可以不具有经济学含义。$A$是因子载荷矩阵。本文采用式($2-3$)中的模型进行宏观经济指标的预测。

对于静态形式的估计,Stock和Watson于1989年最早提出了一种方法:这种方法采用
Kalman 滤波构造似然 函数,并采用极大似然方法来估计参数。
这种方法的 优点是: 在误差项服从正态分布的假定下,能够得到 因子的有效估计量。
然而,由于在估计参数过程中 会应用到非线性数值优化算法,而待估参数的个数与变量维数$p$成比例,
限于当时的计算能力,这种方法只能用于处理低维的精确动态因子模型。

针对维数很高的宏观经济变量,Stock和Watson于2002年给出了一种非参数的方法,并且得到广泛使用。
这种方法使用主成分分析法进行估计,对式($2-3$),因子载荷矩阵$\bm{A}$的估计量$\hat{\bm{A}}$即为
$\bm{X}_t$的样本协方差矩阵$$\hat{\bm{\Sigma}}_{\bm{X}} = \frac1{n}\sum_{t=1}^n\bm{X_t}\bm{X_t}^T
\eqno{(2-4)}$$的前
$r$个最大特征值所对应的特征向量组成的$p\times r$维矩阵,因子的主成分估计量为:
$$ \hat{\bm{F}}_t = \frac1{p}\hat{\bm{A}}\bm{X}_t 
\eqno{(2-5)}$$

主成分估计量具有一系列良好的性质。例如, 在$p \rightarrow \infty$, $n \rightarrow \infty$ 且 $p^2 / n \rightarrow \infty$
的条件下,主成分估计量$\hat{\bm{F}}_t$是因子空间的一致估计量,且在随后的建模 过程中可以当作观测到的数据。

\subsubsection{主成分分析法}
主成分分析法(principal component analaysis, PCA)已经是十分常见的数据降维方法,这里我们不对主成分分析法做详细介绍,仅针对我们的问题,
下面给出使用主成分分分析法计算出估计量的步骤:

记矩阵$\bm{X}_{p\times n}$为$p$维宏观经济变量$\bm{X}_t$在$t_1, ..., t_n$的$n$次观测,

1. 将$\bm{X}$进行中心化;

2. 计算样本协方差矩阵$\frac1{n}\bm{X}\bm{X}^T_{p\times p}$;

3. 对样本协方差矩阵特征值分解,并从小到大排列这些特征值;

4. 因子载荷矩阵估计量$\hat{\bm{A}}_{p\times m} (m < p)$为前$r$个特征值对应的特征向量组成的矩阵。

不难看出,主成分分析法进行估计的关键计算步骤是对样本协方差矩阵的特征值分解。

\subsubsection{奇异值分解} 
这里有必要简要介绍奇异值分解(singular value decomposition, SVD),它也是一类常用的降维方法,在常用统计软件包中
PCA问题的计算往往转化成奇异值分解问题求解。

奇异值分解解决将矩阵$\bm{A}$分解成正交矩阵$\bm{U}$和对角矩阵$\bm{\Sigma}$和另一正交矩阵$\bm{V}^T$的问题,即
$$
    \bm{A}_{m \times n} = \bm{U}_{m \times m}\bm{\Sigma}_{m \times n}\bm{V}_{n \times n}^T
$$

奇异值分解求解的关键就是得到奇异值,而后者是$\bm{A}^T\bm{A}$的特征值的平方根,即奇异值分解的关键计算步骤是对
$\bm{A}^T\bm{A}$进行特征值分解。这里假设我们取$\bm{A} = {\bm{X}^T}/{\sqrt{m}}$,不难看出和PCA问题等价。
因此所有的PCA问题可以转化为SVD进行求解,这样做的原因是高维矩阵进行特征值分解的效率很低,而SVD问题有不需要进行
特征值分解的更高效的迭代计算方法。


\subsection{基于$L_1$范数的因子模型估计}

Stock和Watson提出的主成分估计量等价于通过下面的$L_2$范数优化问题来估计$\bm{A}$和$\bm{F}$,即

$$
\hat{A}_{p\times m}, \hat{F}_{m\times n} = \underset{A,F}{\operatorname{arg\ min} } \|X - A\bm{F}\|_{L_2}
 = \underset{A, \bm{F}}{\operatorname{arg\ min}} \sum_{i=1}^p \sum_{j=1}^m (x_{ij} - a_i^Tf_i)^2 
\eqno{(2-6)}
$$

上述问题不是一个凸优化问题,我们可以转化为等价的PCA问题或者SVD问题解决。但是我们采用的等价解决方法具有
所有的$L_2$范数优化方法普遍的缺陷,那就是对离群值十分敏感。并且PCA和SVD方法都不能够直接处理具有缺失值的数据,
因此对于缺失数据必须进行插补。

然而由于各种原因,高维宏观经济数据中往往具有大量的缺失值和离群值,这就给因子模型的估计带来很多的麻烦,
从而进一步使用因子的估计量进行预测可能会变得不够准确。

\subsubsection{$L_1$范数及其稳健性}
$L_2$范数最优化问题最常见的是最小二乘法。
最小二乘法的优点很多,这里不赘述。尽管在求解大部分问题时用最小二乘估计求解可以得到比较令人满意
的效果, 但最小二乘法也存在一些局限性, 比如, 当收集的数据较少或者具有较多的缺失数据
并且数据中夹杂有异常点时, 用最小二乘法所得的结果就令人难以接受, 在此情况下应用所得到到的回归方程或模型进行预测或者拟合时, 
则预测或拟合的精度是相当低的, 甚至根本不能使用。正因为最小二乘法对数据中的异常值十分敏感,当数据中具有较多离群值时,通过最小二乘、PCA和SVD方法,得到的估计结果也会受到较大影响。

这里可以不使用最小二乘法而选择$L_1$范数的方法即最小一乘法,它可以增加估计的健壮性。
设$\bm{X} = (X_1, X_2, ..., X_p)^T$为一$p$维随机变量,$Y$为响应变量,$\bm{\beta}$为回归系数。
假设我们观察到$i.i.d. $样本$(\bm{X}_1, \bm{X}_2, ..., \bm{X_n})$和$(Y_1, ..., Y_n)$,我们一般使用$\bm{\beta}$
的最小二乘估计量
$$\hat{\bm{\beta}} = \underset{\bm{\beta}}{\operatorname{arg\ min}} \sum_{i=1}^n\|Y_i - \bm{X}^T_i\bm{\beta}\|_{L_2}
=\underset{\bm{\beta}}{\operatorname{arg\ min}} \sum_{i=1}^n(Y_i - \bm{X}^T_i\bm{\beta})^2$$
在稳健统计中,我们经常使用其他的目标函数,例如使用$L_1$范数来代替$L_2$,
$$\hat{\bm{\beta}} = \underset{\bm{\beta}}{\operatorname{arg\ min}} \sum_{i=1}^n\|Y_i - \bm{X}^T_i\bm{\beta}\|_{L_1}
=\underset{\bm{\beta}}{\operatorname{arg\ min}} \sum_{i=1}^n|Y_i - \bm{X}^T_i\bm{\beta}|$$
如图所示,通常使用$L_1$范数在线性回归中可以有效避免离群值造成的干扰。
\begin{figure}[H]
    \centering
    \begin{minipage}[t]{0.48\textwidth}
    \includegraphics[width=8cm]{pics/l1-l2-diff2.pdf}
    \end{minipage}
    \begin{minipage}[t]{0.48\textwidth}
    \includegraphics[width=8cm]{pics/l1-l2-diff.pdf}
    \end{minipage}
    \caption{\small 如图所示,在简单的线性模型的拟合中,出现一个离群值就可以导致最小二乘法拟合出现明显的偏差;
    而含有较多离群值时最小二乘法拟合变得很不可靠;而采用$L_1$范数则具有相当稳健性。}
    \label{fig2.1}
\end{figure}

\subsubsection{使用$L_1$范数}
将式($2-6$)中的目标函数更换为使用$L_1$范数,可得
$$\hat{\bm{A}}_{p\times m}, \hat{\bm{F}}_{m\times n} = \underset{\bm{A},\bm{F}}{\operatorname{arg\ min} } \|X - \bm{A}\bm{F}\|_{L_1}
= \underset{\bm{A}, \bm{F}}{\operatorname{arg\ min}} \sum_{i=1}^p \sum_{j=1}^m |x_{ij} - a_i^Tf_i| \eqno{(2-7)}$$
其中$\|.\|_{L_1}$表示矩阵的$L_1$范数。

式($2-7$)中定义的问题不是一个凸优化问题。但是一旦矩阵$\bm{A}$或者$\bm{F}$固定为常数,那么该问题就成为了一个凸优化问题,
可以找到全局最优解。这启发我们可以使用交替优化的方法求解这个优化问题,即每一步固定$\bm{A}$或者$\bm{F}$的值,来求解另一个
参数。

$$
F^{(t)} = \underset{F}{\operatorname{arg\ min}} \|X - \bm{A}^{(t-1)}\bm{F} \|_{L_1} \eqno{(2-8a)}
$$
$$
\bm{A}^{(t)} = \underset{\bm{A}}{\operatorname{arg\ min}} \|X - \bm{A}F^{T(t)}\|_{L_1} \eqno{(2-8b)}
$$

我们改写式($2-8a$)中的目标函数,
$$
E(F) = \|X - \bm{A}^{(t-1)}\bm{F} \|_{L_1} = \sum_{j=1}^{n}|x_j - \bm{A}^{(t-1)}f_j| \eqno{(2-9)}
$$
其中$x_j$是矩阵$X$的第j列,$f_j$是$\bm{F}$的第j列。于是式($2-8a$)中的问题可以分解为$n$个独立的子优化问题,
求解相应的$f_j$:
$$
    f_j = \underset{\theta}{\operatorname{arg\ min}} |\bm{A}^{(t-1)}\theta - x_j|
    \eqno{(2-10)}
$$
式($2-10$)的全局最优解可以通过求解下面的线性规划问题得到:
$$
    \underset{\theta, t}{\operatorname{min}} 1^T t
$$
$$
    s.t. -t \leq \bm{A}^{(t-1)}\theta - x_j \leq t \eqno{(2-11)}
$$
线性规划问题的计算性能取决于未知变量的个数和和约束的个数。这里的$n$个子优化问题共享了$\bm{A}^{(t-1)}$,
减少了一定的计算量。但是每一个子优化问题都要面临$p$个约束条件和一个
$p$维未知变量,当$x_j$是一个维数非常大的向量时$p$很大),每一个子问题的求解仍然需要大量计算。虽然现在已经
发展出了有效的处理多变量多约束线性规划问题的方法,但是在因子分析的场合下我们仍给出一个性能更好的方法来代替
这里的线性规划。这个方法将在第三章中讨论。

\subsubsection{缺失值处理}
在使用PCA和SVD时我们需要对矩阵$X$进行缺失值插补,然后才能进行计算。
在$L_1$算法中我们不需要进行缺失值插补,在式($2-11$)中,遇到$x_j$具有缺失值的场合,我们直接舍弃相应的
约束条件即可。
我们改写式($2-9$),
$$E(F) = \sum_{i=1}^d \sum_{j=1}^n |x_{ij} - a_i^Tf_j|$$
如果某个项$x_{ij}$缺失,我们直接舍弃目标函数中的对应累加项,在上述算法中对应的做法就是直接删除($2-11$)中的一个约束条件。

\subsubsection{算法步骤}
我们已经发现可以通过交替优化的方法求解式($2-7$)中的优化问题,下面我们更加详细地讨论$L_1$因子分析算法一些细节和具体实现步骤。

(一)初始化

和其他所有的迭代算法一样,在算法开始时,首先我们需要给$\bm{A}$一个初始值$\bm{A}^{(0)}$。对于$\bm{A}$可以采用简单随机数进行初始化,这里我们
为了加快收敛速度,可以使用经过缺失值插补(这里我们使用均值插补)后通过PCA算法进行得到的因子载荷矩阵作为$\bm{A}^{(0)}$。
在本章后续小结的实验中我们可以发现,在含有大量缺失值和离群值的条件下,两种不同的初始化方法最终结果差异并不大。

(二)收敛性

因为目标函数$E(\bm{A}, F) = \|X - \bm{A}\bm{F}\|_{L_1}$在每一个交替的优化步骤中都递减,并且$E(\bm{A},F)$具有下界($\geq 0$)。
因此交替优化算法一定收敛。因此我们可以设定一个收敛域值来停止迭代,这里我们设置终止条件:
    $$ \theta(a_i^{(t)}, a_i^{(t-1)}) <  \alpha $$
这里$\theta(a, b)$表示向量$a$和$b$的夹角;其中$a_i$是$\bm{A}$或者$F$的第i列;$\alpha$是一个很小的正数。

(三)算法步骤

下面我们给出$L_1$因子分析算法的计算步骤:

1.初始化:给出$\bm{A}$,$\Sigma$的初始值$\bm{A}^{(0)}$,$\Sigma^{(0)} = I$,(其中$\Sigma$为一对角矩阵,
$I$为单位矩阵);

2.交替凸优化:对于迭代次数$t = 1, ..., $收敛:
$$F^{(t)} = \underset{F}{\operatorname{arg\ min}} \|X - \bm{A}^{(t-1)}\Sigma^{(t-1)}F^{T}\|_{L_1}$$
$$\bm{A}^{(t)} = \underset{\bm{A}}{\operatorname{arg\ min}} \|X - \bm{A}\Sigma^{(t-1)}F^{T(t)} \|_{L_1}$$
\begin{equation*}
    \text{归一化:}\left\{
                 \begin{array}{clr}
                 N_a = diag(\bm{A}^{(t)T}\bm{A}^{(t)})\\
                 N_f = diag(F^{(t)T}F^{(t)})\\
                 F^{(t)} \leftarrow F^{(t)}N_f^{-1}\\
                 \bm{A}^{(t)}\leftarrow \bm{A}^{(t)}N_a^{-1}\\
                 \Sigma^{(t)} \leftarrow N_a\Sigma^{(t-1)}N_f\\
                 \end{array}
    \right.
\end{equation*}

3.输出结果:$F \leftarrow F\Sigma^{1/2}$,$\bm{A} \leftarrow \bm{A}\Sigma^{1/2}$

\subsection{稳健性实验}
为了检验$L_1$因子分析算法在处理含有大量离群值和缺失值的数据时的稳健性,我们进行模拟实验,来对比采用$L_1$因子分析算法
和PCA因子分析,SVD因子分析以及采用$L_2$目标函数这几种方法的效果。

\subsubsection{数据准备}
为了进行模拟实验,我们首先需要随机产生一个高维低秩的矩阵来模拟高维宏观经济数据集。
我们产生一个$n$维方阵$M$,其中每一个随机元素均服从$[-100, 100]$的均分分布。然后我们对方阵$M$进行奇异值分解
,$M = U\Sigma V^{T}$。假设我们需要产生的低秩矩阵的秩为$r$,则$$X = U_{(:,1:r)}\Sigma_{(1:r,1:r)}V^T_{(:,1:r)}$$
即为我们得到的模拟高维低秩矩阵。

之后我们可以设置一定比例的缺失值和离群值,首先我们在矩阵的左下角剔除部分元素。在剩下的元素中,我们随机选取一部分
然后重新产生随机元素,每个元素服从$[-2000,2000]$上的均匀分布。图2.1展示了一个$30\times30$秩为3的
矩阵在模拟了缺失值和离群值后的情况。

\begin{figure}[H]
    \centering
    \includegraphics[width=.5\textwidth]{pics/matrix.pdf}
    \caption{\small $30\times30$的模拟矩阵,左下角白色代表缺失值,图中灰度越深代表元素的绝对值越大,因此深黑色的点代表了离群值}
    \label{fig2.1}
\end{figure}

令$\hat{\bm{X}} = \hat{\bm{A}}_{30 \times 3}\hat{\bm{F}}_{3 \times 30}$,我们比较因子残差
$\bm{E} = \hat{\bm{X}} -\bm{X}$。我们进行多次实验,取平均结果,观察因子残差的平方的分布情况。


   \section{高维最小绝对值回归的性能研究}

本文的主题是$L_1$范数稳健方法在宏观经济中的应用问题研究。
$L_1$范数是最常用的几种重要范数之一,它在数值计算、统计学、运筹学、
机器学习等领域有着极为重要的应用。在统计学研究中,已经基于$L_1$范数发展出了许多
稳健统计方法,主要是利用$L_1$范数的良好统计性质。

$L_1$范数稳健方法在计量经济学中已有广泛应用,
其中一个重要应用就是最小绝对值回归。最小绝对值回归最早提出是为了弥补最小二乘法的不足。
使用最小二乘法建立的线性回归模型虽然具有较好的解释性,
但是高斯——马尔可夫条件对模型变量的分布有正态性的要求。因此直接使用最小二乘法来建立模型,往往
对原始数据的分布情况要求高。即使变量满足了正态分布,但在模型系数估计时,样本中离群值的存在对
模型的系数估计效果有很大影响。

在宏观经济实证研究中,经济变量一般
不能够服从正态分布,而因受到经济冲击呈现出重尾的特征。对这样的原始数据不能够通过最小二乘法直接建模,
因此常常在回归中剔除异常点,使得经济变量接近正态分布,然而代价是有可能损失较多有价值的信息。

采用最小绝对值回归,是更加稳健的做法。实质上它就是中位数回归,模型的系数也具有很好的解释意义,
已经广泛应用在对居民收入的研究中,并且在金融数据分析领域有着广泛的应用。
在模型的拟合上,最小绝对值回归对离群值不敏感,具有相当的稳健性,因此可以很好处理重尾的宏观经济变量。

高维宏观经济数据,往往包含了维数众多的经济变量,并且样本数量往往非常大,
此时如果想要直接使用最小绝对值回归,在求解时就
需要解决变量维数和约束数量很高的线性规划问题,因此计算的开销很高。

最小绝对值回归在计算上的复杂性在一定程度上限制了它的应用。
一直以来,普遍的加速最小绝对值回归的方法都尝试使用平滑化方法来近似最小绝对值回归的目标函数,
避免直接使用原目标函数。然而,替换目标函数,可能会影响估计量的一致性从而
影响最小绝对值回归模型的解释性。

但是近年来随着统计学和机器学习领域
对$L_1$范数的研究逐渐深入,不断有新的算法尝试解决这个问题。
本章将介绍最小绝对值回归的两种较新的估计算法,一种是基于聚类——迭代拆解的算法;
另一种是基于替代变量的牛顿迭代方法,两种算法基于不同的出发点,
都在一定程度上对解决最小绝对值回归问题做了优化。

\subsection{简介}
\subsubsection{$L_1$范数的稳健性}

$L_2$范数最优化问题最常见的是最小二乘法。
最小二乘法的优点很多,这里不赘述。尽管在求解大部分问题时用最小二乘估计求解可以得到比较令人满意
的效果, 但最小二乘法也存在一些局限性, 比如, 当收集的数据较少或者具有较多的缺失数据
并且数据中夹杂有异常点时, 用最小二乘法所得的结果就令人难以接受, 在此情况下应用所得到到的回归方程或模型进行预测或者拟合时, 
则预测或拟合的精度是相当低的, 甚至根本不能使用。正因为最小二乘法对数据中的异常值十分敏感,当数据中具有较多离群值时,通过最小二乘、PCA和SVD方法,得到的估计结果也会受到较大影响。

这里可以不使用最小二乘法而选择$L_1$范数的方法即最小一乘法,它可以增加估计的健壮性。
设$\bm{X} = (X_1, X_2, ..., X_p)^T$为一$p$维随机变量,$Y$为响应变量,$\bm{\beta}$为回归系数。
假设我们观察到$i.i.d. $样本$\bm{X}_{n\times p} = (\bm{x}_1, \bm{x}_2, ..., \bm{x_n})^T$和$\bm{Y}_{n\times1}=
(y_1, ..., y_n)^T$,我们一般使用$\bm{\beta}$
的最小二乘估计量
\begin{equation}\label{l2loss}
\hat{\bm{\beta}} = \underset{\bm{\beta}}{\operatorname{arg\ min}} \sum_{i=1}^n\|y_i - \bm{x}^T_i\bm{\beta}\|_{L_2}
=\underset{\bm{\beta}}{\operatorname{arg\ min}} \sum_{i=1}^n(y_i - \bm{x}^T_i\bm{\beta})^2
\end{equation}
在稳健统计中,我们经常使用其他的目标函数,例如使用$L_1$范数来代替$L_2$,

\begin{equation}\label{l1loss}
\hat{\bm{\beta}} = \underset{\bm{\beta}}{\operatorname{arg\ min}} \sum_{i=1}^n\|y_i - \bm{x}^T_i\bm{\beta}\|_{L_1}
=\underset{\bm{\beta}}{\operatorname{arg\ min}} \sum_{i=1}^n|y_i - \bm{x}^T_i\bm{\beta}|
\end{equation}
如图所示,通常使用$L_1$范数在线性回归中可以有效避免离群值造成的干扰。
\begin{figure}[H]
    \centering
    \begin{minipage}[t]{0.48\textwidth}
    \includegraphics[width=8cm]{pics/chapter2/l1-l2-diff2.pdf}
    \end{minipage}
    \begin{minipage}[t]{0.48\textwidth}
    \includegraphics[width=8cm]{pics/chapter2/l1-l2-diff.pdf}
    \end{minipage}
    \caption{\small 如图所示,在简单的线性模型的拟合中,出现一个离群值就可以导致最小二乘法拟合出现明显的偏差;
    而含有较多离群值时最小二乘法拟合变得很不可靠;而采用$L_1$范数则具有相当稳健性。}
    \label{fig2.1}

\end{figure}

我们使用$L_1$目标函数来估计$\bm{\beta}$,则其中$\bm{\beta}^*$为最小绝对值回归系数,$e$为一随机噪声,
可以写出最小绝对值回归的一般形式
\begin{equation} \label{l1losstotal}
    \begin{split}
    Y &= \bm{X}^T\bm{\beta}^* + e\\
    \bm{\beta}^* &= \underset{\bm{\beta}\in \mathbb{R}^{p}}{\operatorname {arg\ min}}
    \mathbb{E}|Y - X\bm{\beta}|.
    \end{split}
\end{equation}

\subsubsection{最小绝对值回归的估计方法}
对于\eqref{l1loss},它是一个凸优化问题,但是其不具备显式解,一般求它的数值解。但是其目标函数在$\bm{0}$点不可导,
因此不能直接使用使用梯度下降法,一般来说,该问题
的全局最优解可以通过求解下面的线性规划问题得到:
$$
    \underset{\bm{\beta}, \bm{t}}{\operatorname{min\ }} 1^T \bm{t}
$$
$$
    s.t. -\bm{t} \leq \bm{X}_{n\times p}\bm{\beta}_{p\times1} - \bm{Y}_{n\times 1} \leq \bm{t}
$$
目前对于线性规划问题已经有了比较成熟的解决方法,主要通过单纯形法或者内点法求解,后者的时间复杂度可以控制在多项式时间,
然而,一般而言,当$n$和$p$均很大时,上述线性规划问题面临很高的变量和约束维数,计算速度仍较慢。

由于$L_1$范数的目标函数在机器学习领域的大量使用,已经产生了一些光滑化方法,做法是用一个接近$L_1$的目标函数来替代它,
用来替代的函数往往处处可导,因而可以使用梯度下降法求解。
典型的代表就是使用Huber’s M统计量近似$L_1$范数目标函数,
\begin{equation*}
    \rho(e) = \left\{
        \begin{array}{clr}
            \frac1{2}e^2,\ |e| \leq \gamma \\
            \gamma |e| - \frac1{2}\gamma ^2 ,\ |e| > \gamma
        \end{array}
    \right.
\end{equation*}
其中$\gamma$为某一正数,该问题可以转化为一个二次规划问题求解。

\subsection{聚类——迭代拆解算法}
近年来针对最小绝对值回归的性能研究发展出了除了光滑化目标函数之外的方法,可以在不改变目标函数的情况下,
通过寻求新的优化方法进行求解。这样一来,可不改变最小绝对值回归估计量的统计性质。
这里介绍Park等人于2016年提出的一种基于聚类——迭代拆解算法的最小绝对值回归求解方法。

\subsubsection{聚类——迭代拆解算法说明}
聚类是一种在机器学习中常见的做法,就是按照某种给定的规则,将特征接近的样本点归类到一起。
聚类——迭代拆解算法的提出受到以下事实的启发:
1)优化问题规面临的数据集模庞大,其中许多的样本点在进行参数估计时的贡献是很接近的;
2)如果对相似的样本点进行聚类,提炼该聚类中的信息,避免每个样本点都进入计算,那么就会大大减小问题的规模;
3)假设在聚类后构造的新数据集上不能接近问题的最优解,那么就拆解当前的聚类,在新聚类上进行计算,
聚类个数有限,因此最坏情况下相当于直接求解原问题;

采用聚类——迭代拆解算法求解某个优化问题的前提如下:1)必须能够提出一种规则来对样本点聚类;
2)必须找到合适的聚类和拆解聚类的标准;
3)需要在聚类后构造的新数据集上明确定义新的优化问题;
4)能够判定当前解是否接近最优解。

算法2.1给出了任何一个聚类——迭代拆解算法的主要步骤,注意算法2.1必然在某处停止,
因为每次不断拆解聚类,当聚类个数$|K^{t}|= n$时,相当于计算原问题,此时算法终止。
\begin{table}[H]%%%%%%开始表格
    \centering%把表居中
    \begin{tabular}{{p{0.9\columnwidth}}}%三个c代表该表一共三列,内容全部居中
    
    \toprule%第一道横线 表头
    算法2.1:聚类——迭代拆解算法(Aggregate and Iterative Disaggregate, AID) \\
    \midrule%第二道横线 符号+解释+单位 中间用&隔开
    输入:原始数据集$\bm{X}_{n\times p}$,样本点的下标集合${I} = \{1, 2, ..., n\}$,
    数据的特征下标集合${J} = {1, 2, ..., p}$,原优化问题$P$。\\
    初始化:对原始数据集$\bm{X}$聚类,然后按某种规则产生新的优化数据集$\bm{X}^{1}$。 \\
    对于$t = 1, ..., T$:\\
        记${C}^{t} = \{{C}_1^{t}, ..., C_K^{t}\}$为聚类的集合, $K^t = {1, ..., |K^t|}$为当前聚类的下标,
        \\
        1.根据当前聚类情况${C}^{t}$,构造新的数据集$\bm{X}^{t}$,求解相应的优化问题${P}^{t}$; \\
        2.检查解$\bm{s}^{t}$是否达到最优条件;\\
        3.如果不满足条件,拆解当前聚类。
        \\
    \bottomrule%第三道横线
    \end{tabular}
\end{table}%%%%%%结束表格

\subsubsection{优化最小绝对值回归}
改写\eqref{l1loss}的目标函数,
\begin{equation}\label{l1loss2}
E^* = \underset{\bm{\beta} \in \mathbb{R}^{p}}{\operatorname{min}} 
\sum_{i \in I}|y_i - \sum_{j \in J}x_{ij}\bm{\beta}_j|
\end{equation}

首先给出聚类方法。给定$|K_0|$为目标聚类个数,初始化$C_0 = \{C_1^0, C_2^0, ..., C_K^0\}$,
我们可以使用任意的聚类方法进行初始化。接下来给出如何根据聚类来产生新的数据集,
对于在任一迭代周期内产生的聚类$C^t_k, k = 1, ..., K^t$,取
\begin{equation*}
    x_{kj}^t = \frac{\sum_{i \in C_k^t}x_{ij}}{|C_k^t|},\ j \in J \  
    \text{并且} \
    y_{k}^t = \frac{\sum_{i \in C_k^t}y_i}{|C_k^t|}
\end{equation*}

对于每一个不同的聚类需要给出一个权重来区分信息量不同的聚类,因此在新的数据集上,我们求解下面的问题
\begin{equation}\label{clusterl1}
    F^t =\underset{\bm{\beta}^t \in \mathbb{R}^{p}}{\operatorname{min}} 
    \sum_{k=1}^{K^t}|C_k^t||y_k^t - \sum_{j \in J}x_{kj}^t\bm{\beta}_j^t|
\end{equation}
容易发现,任何\eqref{clusterl1}的可行解都是\eqref{l1loss2}的可行解。
记$\hat{\bm{\beta}^t}$为\eqref{clusterl1}的解,在每次迭代,我们计算此时的原目标函数的取值
\begin{equation}
    E^t = \sum_{i \in I} |y_i - \sum_{j \in J}x_{ij}\hat{\bm{\beta}}_j^t|
\end{equation}

接下来给出拆解聚类的准则:设$t$步的聚类集合为$C^{t}$,该步解为$\hat{\bm{\beta}}_t$,
对于$k = 1, ..., K^t$,计算$\theta_i = y_i - \sum_{j \in J}x_{ij}\hat{\bm{\beta}}_j^t$,
1)若对于任意$i \in C_k^t$,$\theta_i$有相同的符号,那么该聚类$C^t_k$将保留到下次迭代,即$C^{t+1} \leftarrow C^{t+1}\bigcup C^t_k$,见图
\ref{aid-demo};

2)若不满足上述条件,那么根据$\theta_i$符号异同,将$C^t_k$分成两个集合,$C_{k+}^t = \{i \in C_k^t | \theta_i > 0\}$ ,
$C_{k-}^t = \{i \in C_k^t | \theta_i < 0\}$,见图\ref{aid-demo}b,这两个集合在下一步形成新的聚类,
即$C^{t+1} \leftarrow C^{t+1}\bigcup \{ C_{k+}^t, C_{k-}^t\}$,见图\ref{aid-demo}c。

\begin{figure}[H]
    \centering
    \begin{subfigure}[t]{0.3\textwidth}\label{aid-demo1}
    \includegraphics[width=6cm]{pics/chapter2/aid-demo-a.pdf}
    \captionof{figure}{}
    \end{subfigure}
    \begin{subfigure}[t]{0.3\textwidth}\label{aid-demo2}
    \includegraphics[width=6cm]{pics/chapter2/aid-demo-b.pdf}
    \captionof{figure}{}
    \end{subfigure}
    \begin{subfigure}[t]{0.3\textwidth}\label{aid-demo3}
    \includegraphics[width=6cm]{pics/chapter2/aid-demo-c.pdf}
    \captionof{figure}{}
    \end{subfigure}
    \caption{\small 聚类拆解步骤示意图}
    \label{aid-demo}

\end{figure}

结合算法2.1,到这里已经给出了完成聚类——迭代拆解最小绝对值回归的所有计算步骤,当聚类无法继续划分时,迭代终止。

下面证明最后一次迭代的解$\hat{\bm{\beta}}_j^T$就是\eqref{l1loss2}的解$\bm{\beta}^*$,
\begin{equation*}
    \begin{split}
        E^* & = \sum_{i \in I} |y_i - \sum_{j \in J}x_{ij}\bm{\beta}j^*|
        = \sum_{k \in K_t}\sum_{i \in C_k^t}|y_i - \sum_{j \in J}x_{ij}\beta_j^*| \\
        & \geq \sum_{k \in K^t}|\sum_{i\in C_k^t}y_i - \sum_{j \in J}x_{ij}\bm{\beta}_j^*|
        = \sum_{k \in K^t}|C_k^t||y_k^t - \sum_{j \in J}x_{kj}^t\bm{\beta}_j^*|\\
        & \geq \sum_{k \in K^t} |C_k^t||y_K^t - \sum_{j\in J}x_{kj} \hat{\bm{\beta}}_j^t|
        = \sum_{k \in K^t} |\sum_{i \in C_k^i} yi - \sum_{i \in C_k^t}\sum_{j \in J}x_{ij}\hat{\bm{\beta}}_j^t| \\
        & = \sum_{k \in k^t} \sum{i \in C_k^i}|y_i - \sum_{j \in J}x_{ij}\hat{\bm{\beta}}_j^t|
        = \sum_{i \in I}|y_i - \sum_{j \in J} x_{ij} \hat{\bm{\beta}}_j^t| 
        = E^t
    \end{split}
\end{equation*}

因为$\hat{\bm{\beta}}_t$是\eqref{l1loss2}的可行解,又显然$E^* \leq E^t$,
这就证明了$E^* = E^t$,注意到$ \sum_{k \in K^t} |C_k^t||y_K^t - \sum_{j\in J}x_{kj} \hat{\bm{\beta}}_j^t|$
就是$F^t$,因此$E^t = F^t$。因此最后一次迭代$F^T$的最优解$\hat{\bm{\beta}}_j^T$就是原问题的最优解。

\subsubsection{优化高维最小绝对值回归}
前面给出的算法已经在很大程度上优化了在处理高维宏观经济变量时,最小绝对值回归计算性能的问题。
而在高维宏观经济实证研究中,变量的维数众多,在建立最小绝对值回归模型时,还需要考虑到变量筛选问题。

对于\eqref{l1losstotal},为了进行变量的筛选,我们需要对估计量$\bm{\beta}^*$的维数做出约束,
一般来说,我们通常采用以下3三种方法:1)$\|\bm{\beta}\|_0 = p$,可以直接选择入选变量的个数;
2)$\|\bm{\beta}\|_2 < s$,该法又称为岭回归;3)$\|\bm{\beta}\|_1 < s$,即为常用的$L_1$正则化方法。
通过加入新的约束,问题的形式也发生了变化。

Park等人于2019年的研究中给出了进一步的结论,对于问题
\begin{equation}\label{l1conclusion}
    E^* = \underset{B\in \phi}{\operatorname{min}} ||Y - f(X, B)||_{L_1}
\end{equation}
其中$Y$为模型响应变量数据矩阵,$X$为解释变量数据矩阵,$B$为模型系数矩阵,$\phi$为模型系数的约束条件。
$f$为任一目标函数,若$f$满足条件
\begin{equation}\label{fcondition}
    f(B, WX) = Wf(B, X)
\end{equation}
其中$W$为一权重矩阵。那么就可以通过聚类——迭代拆解算法求原优化问题的最优解。

而在岭回归和LASSO中,都通过给目标函数加上惩罚项来进行求解,岭回归对应的惩罚项$s\sum_{i=1}^p\bm{\beta}_i^2$,
$L_1$正则化对应的惩罚项为$s\sum_{i=1}^p|\bm{\beta}_i|$,容易发现,加入惩罚项后目标函数仍然满足\eqref{fcondition}
。因此,对于需要进行变量筛选的高维最小绝对值回归问题,我们可以修改目标函数,通过算法2.1解决。

\subsection{一种基于替代变量的估计方法}
聚类——迭代拆解算法是通过有效减少$L_1$目标函数最小化的问题规模来提升求解性能,
每一次迭代仍需要通过线性规划方法求解$L_1$目标函数的最小化问题。
Weidong Liu等人于2020年提出了一种基于替代变量的估计方法,通过替代变量将分位数损失函数转化为$L_2$损失函数,
避免使用线性规划求解分位数损失,因此可以显著提升一般分位数回归的求解性能。

考虑到最小绝对值回归为其特殊情形,且该估计量具有良好的性质,可以在一定程度上解决最小绝对值回归的性能问题。
下面我们介绍如何将该估计应用在最小绝对值回归的场景下,并且给出具体的计算步骤。

\subsubsection{基于替代变量的迭代算法}
考虑\eqref{l1losstotal},
对目标函数稍作形式变换,其中$\rho(x) = x(0.5 - \mathbb{I}[x \leq 0])$,
$\mathbb{I}(x)$为指示函数。
\begin{equation}
\bm{\beta}^* = \underset{\bm{\beta} \in \mathbb{R}^{p}}{\operatorname{arg\ min}}\mathbb{E}|Y - \bm{X}^T\bm{\beta}| = 
\underset{\bm{\beta} \in \mathbb{R}^{p}}{\operatorname{arg\ min}}\mathbb{E}\rho(Y - \bm{X}^T\bm{\beta})
\end{equation}
若已知n\ $i.i.d.$ 的样本$(\bm{X}_i, Y_i)\ (1 \leq i \leq n)$,令$\hat{\bm{\beta}}$为$\bm{\beta}^*$的估计量,则
\begin{equation}\label{rho-problem}
    \hat{\bm{\beta}} = \underset{\bm{\beta} \in \mathbb{R}^{p}}{\operatorname{arg\ min}}\frac1{n}\sum_{i=1}^{n}\rho(Y_i - \bm{X}_i^T\bm{\beta})
\end{equation}

一般地,我们使用牛顿迭代法求解某随机优化问题:
\begin{equation}
    \bm{\beta}^* = \underset{\bm{\beta} \in \mathbb{R}^{p+1}}{\operatorname{arg\ min}} \mathbb{E}[G(\bm{\beta};\bm{X},Y)]
\end{equation}
其中$G(\bm{\beta};\bm{X}, Y)$是损失函数,$\bm{X}$和$Y$分别是$p+1$维自变量和一元响应变量,$\bm{\beta}$为回归系数。使用牛顿-拉弗森迭代来求解,
\begin{equation}
    \tilde{\bm{\beta}}_1 = \bm{\beta}_0 - \bm{H}(\bm{\beta}_0)^{-1}\mathbb{E}[g(\bm{\beta};\bm{X},Y)]
\end{equation}
其中$\bm{\beta}_0$是一个初始估计,$g(\bm{\beta};\bm{X},Y)$为损失函数$G(\bm{\beta};\bm{X},Y)$关于$\bm{\beta}$的梯度。\\
$\bm{H}(\bm{\beta}):=\partial\mathbb{E}[g(\bm{\beta};\bm{X},Y)
/\partial\bm{\beta}$表示$\mathbb{E}G(\bm{\beta};\bm{X},Y)$的海赛矩阵。特别地,我们这里考虑损失函数为$L_1$损失的特殊情形,
\begin{equation}\label{rho-condition}
    G(\bm{\beta};\bm{X},Y) = \rho(Y - \bm{X}^T\bm{\beta})
\end{equation}

在\eqref{rho-condition}的条件下,$g(\bm{\beta};\bm{X},Y) = \bm{X}(\mathbb{I}[Y - \bm{X}^T\bm{\beta} < 0] - 0.5)$。\\
并且,$\bm{H}(\bm{\beta}) = \mathbb{E}(\bm{X}\bm{X}^Tf(\bm{X}^T(\bm{\beta} - \bm{\beta}^*)))$,这里
$f(x)$是噪声$e$的密度函数。当初始估计量$\bm{\beta}_0$和$\bm{\beta}^*$很接近时,$\bm{H}(\bm{\beta}_0)$就会很接近
$\bm{H}(\bm{\beta}^*) = \bm{\Sigma}f(0)$,这里$\bm{\Sigma} = \mathbb{E}\bm{X}\bm{X}^T$是$\bm{X}$的协方差
矩阵。使用$\bm{H}(\bm{\beta}^*)$替换$\bm{H}(\bm{\beta}_0)$,可得

\begin{equation}\label{rho-beta1}
    \bm{\beta}_1 = \bm{\beta}_0 - \bm{H}(\bm{\beta}^*)^{-1}\mathbb{E}[g(\bm{\beta};\bm{X}, Y)]
    = \bm{\beta}_0 - \bm{\Sigma}^{-1}f^{-1}(0)\mathbb{E}[g(\bm{\beta}_0;\bm{X},Y)]
\end{equation}
在$\bm{\beta}^*$对$\mathbb{E}[g(\bm{\beta}_0;\bm{X},Y)$进行泰勒展开,
\begin{equation*}
    \begin{split}
\mathbb{E}[g(\bm{\beta}_0;\bm{X},Y) &= \bm{H}(\bm{\beta}^*)(\bm{\beta}_0 - \bm{\beta}^*) + O(|\bm{\beta}_0 - \bm{\beta}^*|_2^2) \\
 &= \bm{\Sigma}f(0)(\bm{\beta}_0 - \bm{\beta}^*) + O(|\bm{\beta}_0 - \bm{\beta}^*|_2^2)
    \end{split}
\end{equation*}
结合\eqref{rho-beta1},可以得到
\begin{equation*}
    \begin{split}
        |\bm{\beta}_1 - \bm{\beta}^*|_2 &=  |\bm{\beta}_0 - \bm{\Sigma}^{-1}f^{-1}(0)(
            \bm{\Sigma}f(0)(\bm{\beta}_0 - \bm{\beta}^*) + O(|\bm{\beta}_0 - \bm{\beta}^*|_2^2)
        ) - \bm{\beta}^*|_2\\
        &= O(|\bm{\beta}_0 - \bm{\beta}^*|_2^2)
    \end{split}
\end{equation*}
因此,如果我们得到一个$\bm{\beta}^*$的一致估计量$\bm{\beta}_0$,我们就可以通过\eqref{rho-beta1}得到
偏误更小的估计。

下面我们将\eqref{rho-beta1}转化成一个最小二乘问题。首先我们重写该式,
\begin{equation}
    \begin{split}
    \bm{\beta}_1 &= \bm{\Sigma}^{-1}(\bm{\Sigma}\bm{\beta}_0 - f^{-1}(0)\mathbb{E}[g(\bm{\beta}_0;\bm{X},Y)])\\
    &= \bm{\Sigma}^{-1}\mathbb{E}[\bm{X}\{\bm{X}^T\bm{\beta}_0 - f^{-1}(0)(\mathbb{I}[Y \leq \bm{X}^T\bm{\beta}_0] - 0.5)\}]
    \end{split}
\end{equation}
这里我们定义一个新的响应变量$\tilde Y$,
\begin{equation}
    \tilde Y = \bm{X}^T\bm{\beta}_0 - f^{-1}(0) (\mathbb{I}[Y \leq \bm{X}^T\bm{\beta}_0] - 0.5)
\end{equation}
那么$\bm{\beta}_1 = \bm{\Sigma}^{-1}\mathbb{E}(\bm{X}\tilde{Y})$就是线性回归问题$\tilde Y = \bm{X}^T\bm{\beta}$
的最优回归系数,即
\begin{equation}\label{beta1-original}
    \bm{\beta}_1 = \underset{\bm{\beta} \in \mathbb{R}^{p}}{\operatorname{arg\ min}} 
    \mathbb{E}(\tilde Y - \bm{X}^T \bm{\beta})^2
\end{equation}
给定$i.i.d.$样本$(\bm{X}_i, Y_i)$,构造
\begin{equation}\label{svny}
    \tilde{Y}_i = \bm{X}_i^T\hat{\bm{\beta}}_0 - \hat{f}^{-1}(0)
    (\mathbb{I}[Y_i \leq \bm{X}_i^T \hat{\bm{\beta}}_0] - 0.5)
\end{equation}
其中$\hat{\bm{\beta}}_0$为$\bm{\beta}^*$的一个初始估计,$\hat{f}(0)$为$f(0)$的一个估计
\begin{equation}\label{yl2loss}
    \hat{\bm{\beta}} = \underset{\bm{\beta} \in \mathbb{R}^{p}}{\operatorname{arg\ min}}
    \frac1{n} \sum_{i=1}^n(\tilde{Y}_i - \bm{X}_i^T\bm{\beta})^2
\end{equation}
这里我们选择某估计量作为$\hat{\bm{\beta}}_0$,并且采用$f(0)$的核密度估计作为$\hat{f}(0)$,

$$
    \hat{f}(0) = \frac1{nh}\sum_{i=1}^nK(\frac{Y_i - \bm{X}^T_i\hat{\bm{\beta}}_0}{h})
$$
其中$K(x)$为核函数,$h \rightarrow 0$是带宽,本方法对待估系数的稀疏性有较强的要求(Weidong Liu,2020),并且
在每次迭代都需要调整带宽。

只要给定的初始值$\hat{\bm{\beta}}_0$是$\bm{\beta}^*$的一致估计量,那么$\hat{\bm{\beta}}$就将会是一个更加接近
$\bm{\beta}^*$的新的估计,并且\eqref{yl2loss}为最小二乘问题,其计算十分简便。

我们当然可以将$\hat{\bm{\beta}}$作为\eqref{rho-beta1}的初始值,这样继续构造替代变量进行迭代,最终收敛到$\bm{\beta}^*$。
算法2给出了使用替代变量估计方法的计算步骤。
\begin{table}[H]%%%%%%开始表格
    \centering%把表居中
    \begin{tabular}{{p{0.9\columnwidth}}}%三个c代表该表一共三列,内容全部居中
    
    \toprule%第一道横线 表头
    算法2.2 使用替代变量迭代算法方法求解最小绝对值回归问题(SVN, substitue variable newton method)\\
    \midrule%第二道横线 符号+解释+单位 中间用&隔开
    输入:$Y$和$\bm{X}$的样本$\bm{Y} = (Y_1, Y2, ..., Y_n)$,$\bm{X} = (\bm{X}^T_1, \bm{X}^T_2, ..., \bm{X}^T_n)$,
    迭代次数$T$,核函数$K$,依赖于迭代次数的带宽$h_t(t = 1, ..., T)。$
    \\
    初始化:给出初始相合估计量,$\hat{\bm{\beta}}^{(0)} $,将样本划分成$J$个均等子集,样本量均为$m$,为了给出最小绝对值回归估计量的相合估计,
    我们取任一子集里面的数据,直接使用最小一乘法估计$\hat{\bm{\beta}}_{(0)}$。
    \\
    对于$t = 1, ..., T$:
    取新的数据子集,在该数据集上
    \\
        1. 计算$\hat{f}^{t}(0)$,
        $$
        \hat{f}^{t}(0) = \frac1{mh}\sum_{i=1}^{m}K(\frac{Y_i - \bm{X}_i^T\hat{\bm{\beta}}^{t-1}}{h_t})
        $$
    \\
        2. 计算$\tilde{Y} = (\tilde Y_1, \tilde Y_2, ..., \tilde Y_m)$,
        $$
        \tilde{Y}_i = \bm{X}^T_i\hat{\bm{\beta}}^{t-1} - \hat{f}^{t}(0)^{-1}
        (\mathbb{I}[Y_i \leq \bm{X}_i^T \hat{\bm{\beta}}^{t-1}] - 0.5)
        $$
    \\
        3. 计算$\hat{\bm{\beta}}^{t}$,
        $$
        \hat{\bm{\beta}}^{t} = \underset{\bm{\beta} \in \mathbb{R}^{p}}{\operatorname{art\ min}}
        \frac1{m}\sum_{i=1}^m (\tilde{Y}_i - \bm{X}_i^T\bm{\beta})^2
        $$
    \\
    输出:$\bm{\beta}^{(T)}$
    \\
    \bottomrule%第三道横线
    \end{tabular}
\end{table}%%%%%%结束表格

\subsubsection{优化高维最小绝对值回归}
在高维相依自变量的情形下,我们只需在\eqref{beta1-original}加入惩罚项即可,例如我们使用$L_1$正则化,那么
\begin{equation}
    \bm{\beta}_{1, \lambda} =\underset{\bm{\beta} \in \mathbb{R}^{p}}{\operatorname{arg\ min}} 
    \mathbb{E}(\tilde Y - \bm{X}^T \bm{\beta})^2 + \lambda|\bm{\beta}|
\end{equation}
对应的,最终估计量计算如下
\begin{equation}\label{beta1-lasso}
    \hat{\bm{\beta}} = \underset{\bm{\beta} \in \mathbb{R}^{p}}{\operatorname{arg\ min}}
    \frac1{n} \sum_{i=1}^n(\tilde{Y}_i - \bm{X}_i^T\bm{\beta})^2 + \lambda|\bm{\beta}|
\end{equation}
\eqref{beta1-lasso}是著名的LASSO问题,已经有了快速解决的算法。因此,对于高维情形,算法2仅需稍作改动。

\subsection{模拟实验}
本章前面两节介绍了聚类——迭代拆解算法和一种基于替代变量的迭代算法,
前者通过减小问题规模、逐步逼近最优解的方法来提升计算性能,
而后者通过变量替换的方法将最小绝对值回归问题转化为最小二乘问题,在给定一个相合估计条件下逼近最优解。
两种方法都可以处理带有惩罚项和不带有惩罚项的最小绝对值回归问题,并且我们已经给出了各自计算的具体步骤。

本节将通过一个数值模拟实验来比较和分析两种方法的性能表现,分析其优缺点。

\subsubsection{数据准备}
设定模拟数据来自下面的模型:
\begin{equation*}
    {Y}_i = \bm{X}_i^T \bm{\beta} + e_i, i = 1, 2, ..., n
\end{equation*}
其中$\bm{X}_i = (1, X_{i,1}, ..., X_{i, p-1})$为$p$维随机向量,
在比较不带惩罚项的最小绝对值回归的场合,令$X_i$的各维度服从独立同分布。
我们设置不同$p, n$的组合用来测试算法的计算性能,并观察在不同的噪声分布下,估计结果的准确性。
对于系数$\bm{\beta}$,设$s$为其$L_0$范数,取
$$
    \bm{\beta} = (\frac{10}{s}, \frac{20}{s}, ..., \frac{10(s-1)}{s}, 10, 0, ..., 0)
$$

对于算法2.1,以下简称SVN的每步最佳带宽依赖于数据集的性质,参考weidong liu等,这里给出
$$
    h^t = \sqrt{\frac{s\log n}{n}} + s^{-1/2} (\frac{s^2\log n}{10m})^{(t+1)/2},
$$
核函数选取高斯核函数。

对于聚类——迭代拆解算法,我们这里选择初始聚类方法如下:
首先从原始数据中少量采样(本实验取$max\{0.5\%n, m\}$),通过最小绝对值回归给出系数估计$\bm{\beta}^{init}$,
对每一个原始数据点,计算其在当前模型系数下的残差,然后进行K-means聚类得到$C^0$。

本次实验对比基准使用线性规划内点法(以下简称LP)求解。我们给出了SVN算法和AID算法的Python3实现,数值计算基于numpy包,
作为基准的LP使用Scipy数值计算包的对应实现。

\subsubsection{实验结果}

我们采用最终估计值和真实值差的$L_2$范数来衡量准确性。
在实验结果中中我们着重观察算法的收敛性、算法的准确性和计算性能。

我们在高斯噪声下,比较三种算法的性能表现。

表\ref{tab-performance}中,$r^0$表示AID算法初始聚类$K_0/n$的值,而$r^T$表示算法终止时$K_T/n$的取值。
需要注意的是,对于AID算法,$T$表示算法终止时经历的迭代次数,$Time$表示算法终止时运算的cpu时间。
而对于SVN算法,$T$表示其每估计值达到稳定的迭代次数,$Time$表示全部数据参与迭代完毕经历的cpu时间。
\begin{table}[H]
    \centering
    \begin{tabular}{@{}ccccccccc@{}}
    \toprule
           &     & \multicolumn{4}{c}{AID}        & \multicolumn{2}{c}{SVN} & LP        \\ \midrule
    n      & m   & $r^0$(\%) & $r^T$(\%) & T  & Time(sec) & T      & Time(sec)      & Time(sec) \\ \midrule
    20000  & 10  & 0.05  & 2.90  & 6  & 2         &  3      &         0.38       & 0.20      \\
    20000  & 50 & 0.15  & 12.00 & 12 & 13        &   7     &          6     & 3       \\
    20000  & 100 & 0.15  & 12.00 & 12 & 38        &  11      &         27    & 21         \\
    20000  & 200 & 0.66  & 21.55 & 9  & 118        &   8     &          76      & 126        \\
    20000  & 500 & 0.80  & 28.00 & 11 &  535      &  3      &              322  &     813 \\ 
    40000  & 10  & 0.05  & 2.90  & 6  & 3        &  3      &         0.74      & 0.23     \\
    40000  & 50 & 0.15  & 12.00 & 12 & 31        &   7     &          14     & 7       \\
    40000  & 100 & 0.15  & 12.00 & 12 & 84       &  8      &         56    & 48         \\
    40000  & 200 & 0.66  & 21.55 & 9  & 210        &   8     &          154     & 258        \\
    40000  & 500 & 0.80  & 28.00 & 11 & 1965       &  3      &             1143   &  3446      \\ 
    200000  & 100 & 0.80  & 11.00 & 11 & 134       &  3      &          361      &   496     \\ 
    200000  & 200 & 0.80  & 28.00 & 11 & 2476       &  3      &          1631      &   3329     \\ 
    \bottomrule
    \end{tabular}
    \caption{\small 高斯噪声下三种估计算法的性能比较。(处理机情况:苹果M1芯片8核,8GB内存,OSX,CPython解释器3.82Arm版)}
    \label{tab-performance}
\end{table}

可以看出在$n,m$较大场合下,AID算法和SVN算法均在性能上有领先,尤其是SVN算法具有优异的性能表现。

\begin{figure}[H]
    \centering
    \begin{subfigure}[t]{0.3\textwidth}\label{svn-demo1}
    \includegraphics[width=5cm]{pics/chapter2/svn-l2.pdf}
    \captionof{figure}{}
    \end{subfigure}
    \begin{subfigure}[t]{0.3\textwidth}\label{svn-demo2}
    \includegraphics[width=5cm]{pics/chapter2/svn-l2-3.pdf}
    \captionof{figure}{}
    \end{subfigure}
    \begin{subfigure}[t]{0.3\textwidth}\label{svn-demo3}
    \includegraphics[width=5cm]{pics/chapter2/svn-conver.pdf}
    \captionof{figure}{}
    \end{subfigure}
    \caption{ \small SVN算法的迭代情况图。横轴代表了迭代次数,纵轴代表了估计的误差平方和。(a)中$n=20000,p=10,s=4$,(b)中$n=20000,p=50,s=20$,
    (c)中$n=20000,p=100,s=80$。}
    \label{svn-demo}

\end{figure}

图\ref{svn-demo}展示了SVN算法出色的收敛速度。我们观察到SVN算法在系数具有稀疏性时表现明显更好,这也符合该算法提出的理论基础。
我们可发现,在稀疏条件下,SVN算法在很快趋近平稳;在系数不具有系数性时,SVN算法准确度明显降低,并且收敛速度变慢。
然而大多数的情况下,对于常见的宏观经济线性模型,系数稀疏的现象更加普遍。在稀疏的条件下,SVN算法表现出了十分出色的性能和估计精确度,需要
注意的是在样本充分大的情况下,我们不需要将划分的每一组数据纳入计算,在能够使得估计量取值稳定的有限步骤内返回即可,这时SVN算法的计算时间还可以进一步缩短。

接下来我们观察不同噪声分布的影响,由于AID算法和线性规划方法具有同样的解,图\ref{svn-noise}给出了不同的噪声分布下SVN算法的表现,
可以看到,由于SVN算法并没有直接使用$L_1$损失,其在面对柯西噪声这种重尾的极端情况下精度不如直接优化$L_1$损失的算法。

\begin{figure}[H]
    \centering
    \begin{subfigure}[t]{0.3\textwidth}\label{svn-demo1}
    \includegraphics[width=5cm]{pics/chapter2/gauss.pdf}
    \captionof{figure}{}
    \end{subfigure}
    \begin{subfigure}[t]{0.3\textwidth}\label{svn-demo2}
    \includegraphics[width=5cm]{pics/chapter2/expential.pdf}
    \captionof{figure}{}
    \end{subfigure}
    \begin{subfigure}[t]{0.3\textwidth}\label{svn-demo3}
    \includegraphics[width=5cm]{pics/chapter2/cauchy.pdf}
    \captionof{figure}{}
    \end{subfigure}
    \caption{ \small SVN算法在各种噪声下的表现。$n=20000,p=50,s=20$}
    \label{svn-noise}

\end{figure}

而AID算法虽然在计算时间上并没有SVN算法表现出色,但是它的确在大规模数据下减少了计算量。并且AID算法可以保证获得最优解,
这是SVN算法不具备的优势;另外,SVN算法的准确度和带估系数的稀疏性紧密相关,在高维非稀疏的场合下,SVN算法并不具备很好的准确性。

考虑到在加入惩罚项后,AID算法和LP需要解决增加该额外约束的线性规划问题,其计算复杂度将上升。
而SVN算法,需要求解\eqref{beta1-lasso},目前已有的LASSO算法可以快速求解。
因此SVN算法在带有$L_1$惩罚项的场景下,更加体现其性能优势,因此本节不再赘述。


\subsection{本章小结}

本章首先简要介绍了$L_1$范数的概念,列举了一些其在经济数据分析领域的应用。随后,讨论了
最小绝对值回归的稳健性和其求解方法。

最小绝对值回归已经广泛应用在高维宏观经济数据的处理中,但是它在应用中存在一定的性能问题。
本章介绍了两种可以应用于最小绝对值回归的优化手段:聚类——迭代拆解算法和基于替代变量的估计方法。
进行了数值模拟实验,分析了其优缺点,并给出了其在实际应用中的几点建议。

聚类——迭代拆解算法
是一种非常好的思想,它结合了机器学习中聚类的方法,聚类的作用可以将信息进行提纯,在经过处理
后的数据集上解决原始问题对应的加权优化问题,可以避免直接解决原始问题而耗费庞大的计算量。
聚类——迭代拆解算法可以应用于解决最小绝对值回归问题,它在一定程度上减少了线性规划的计算量,
并且在理论上具有最优解,即它的解一定可以和直接对所有数据求解进行线性规划问题一样好。

另外,本章介绍了另外一种巧妙的算法——一种基于替代变量的迭代算法,该算法通过
构造替代变量,从而将原来需要通过线性规划解决的最小绝对值回归问题转化为
解决替代变量的最小二乘问题。这样一来,计算的复杂度就大大降低。该方法在
实际系数具有系数性时表现十分优秀。

这两种算法都可以在特定的条件下,优化最小绝对值回归的计算速度。
目前高维宏观经济数据分析中也经常使用最小绝对值回归进行某些计算,
我们可以根据数据的规模和问题的特点来选择以上两种方法来代替线性规划。

最后,我们在本章对最小绝对值回归性能优化的研究,会对之后
章节对$L_1$范数主成分分析的应用研究有推动作用。

   \section{$L_1$范数主成分分析算法的性能研究}

\subsection{基于$L_1$范数最大化算法}

\subsection{基于$L_1$范数最小化算法}

\subsection{模拟实验}
   \section{总结与展望}
本文研究了$L_1$主成分分析其在宏观经济因子模型中的应用。

\subsection{研究成果总结}

\subsection{研究不足}
我们虽然已经对求解$L_1$主成分分析的交替凸优化算法进行了优化尝试,
但是在规模很大的数据场景下,其
计算时间仍然不可接受。
在数据满足因子模型时,我们的优化尝试更加有效些,然而我们的方法依旧不易处理高维数据($p >> n$)的问题。

\subsection{研究展望}
自$L_1$主成分分析提出,不断有新的求解
算法被开发出来,然而至今没有算法可以求解问题$P_3$的最优解,并且维数和观测都很大的$P_4$问题已经被证明
为NP难题。
因此,继续深入探究$P_3$的求解算法,推广$L_1$范数主成分分析的应用就成了我们下一步的
研究愿景。

国外学者已经在实证研究中大量应用因子模型,其中包括了应用于混频数据和高维数据进行经济短期预测和
实时预测的研究。我们希望在后续的研究中,能够涉及因子模型更加丰富的应用场景,
对$L_1$主成分估计的应用价值做更多的发掘。

最后,我们知道$L_1$范数主成分分析问题的两种形式对应了两种不同的解,而本文并未对最优解的统计性质
做理论上的探究。在后续研究中,深入探讨相关理论也将是重要的目标。


   \addcontentsline{toc}{section}{参考文献}%将“参考文献加入目录中”
   \bibliography{refs}
\end{document}