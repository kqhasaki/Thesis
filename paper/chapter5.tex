\section{总结与展望}
本文研究了$L_1$主成分分析其在宏观经济因子模型中的应用。

\subsection{研究成果总结}

\subsection{研究不足}
我们虽然已经对求解$L_1$主成分分析的交替凸优化算法进行了优化尝试,
但是在规模很大的数据场景下,其
计算时间仍然不可接受。
在数据满足因子模型时,我们的优化尝试更加有效些,然而我们的方法依旧不易处理高维数据($p >> n$)的问题。

\subsection{研究展望}
自$L_1$主成分分析提出,不断有新的求解
算法被开发出来,然而至今没有算法可以求解问题$P_3$的最优解,并且维数和观测都很大的$P_4$问题已经被证明
为NP难题。
因此,继续深入探究$P_3$的求解算法,推广$L_1$范数主成分分析的应用就成了我们下一步的
研究愿景。

国外学者已经在实证研究中大量应用因子模型,其中包括了应用于混频数据和高维数据进行经济短期预测和
实时预测的研究。我们希望在后续的研究中,能够涉及因子模型更加丰富的应用场景,
对$L_1$主成分估计的应用价值做更多的发掘。

最后,我们知道$L_1$范数主成分分析问题的两种形式对应了两种不同的解,而本文并未对最优解的统计性质
做理论上的探究。在后续研究中,深入探讨相关理论也将是重要的目标。

