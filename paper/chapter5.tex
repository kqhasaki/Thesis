\section{总结与展望}


\subsection{研究成果总结}
$L_1$范数主成分分析比传统主成分分析更加稳健可靠,使用$L_1$主成分分析可以进行更加稳健的因子模型估计
从而更好地进行宏观经济分析和预测。本文围绕相关问题展开了理论与实证研究,并取得如下成果:

1)扩散指数模型可以利用高维宏观经济变量的静态因子作为预测变量来进行有效的宏观经济预测。
本文创新性地将$L_1$主成分分析应用于静态因子的估计中,实证研究表明:$L_1$主成分分析估计得到的因子
具有良好的预测效果,并且更具稳健性和准确性。

2)交替凸优化算法是求解$L_1$主成分分析的经典算法,然而其在计算效率上较慢。本文
首先研究了最小绝对值回归的两种优化求解算法,之后我们将
优化的最小绝对值回归求解算法应用于交替凸优化算法中,代替线性规划,获得了更高的计算效率。

3)最后,$L_1$主成分分析根据问题不同有不同的求解方式。我们对两种不同求解算法都进行了介绍,并通过
模拟与实证研究,论证了两种方法都可以对因子模型进行稳健的估计,在宏观经济预测中更具稳健性和准确性。

\subsection{不足与展望}
我们虽然已经对求解$L_1$主成分分析的交替凸优化算法进行了优化尝试,
但是在规模很大的数据场景下,其
计算时间仍然不可接受。
在数据满足因子模型时,我们的优化尝试更加有效些,然而我们的方法依旧不易处理高维数据($p >> n$)的问题。

自$L_1$主成分分析提出,不断有新的求解
算法被开发出来,然而至今没有算法可以求解问题$P_3$的最优解,并且维数和观测都很大的$P_4$问题已经被证明
为NP难题。
因此,继续深入探究$P_3$的求解算法,推广$L_1$范数主成分分析的应用就成了我们下一步的
研究愿景。

国外学者已经在实证研究中大量应用因子模型,其中包括了应用于混频数据和高维数据进行经济短期预测和
实时预测的研究。我们希望在后续的研究中,能够涉及因子模型更加丰富的应用场景,
对$L_1$主成分估计的应用价值做更多的发掘。

最后,我们知道$L_1$范数主成分分析问题的两种形式对应了两种不同的解,而本文并未对最优解的统计性质
做理论上的探究。在后续研究中,深入探讨相关理论也将是重要的目标。

