\centerline{\large\heiti 后\ 记}

\vspace{2ex}

时光荏苒,我已在南京审计大学度过了三个年头,在统计与数学学院的研究生生涯也已接近尾声。
在南京审计大学的这段读书时光,注定会是我一生中的宝贵记忆。

在毕业论文完成之际,首先要感谢我的导师孔新兵教授。跟老师相处的三年时光下来,我被老师
渊博的专业知识、严谨的治学风格和平易近人的学者风范所深深感染。毕业论文的顺利完成,离不开
老师的悉心指导和耐心修改,从选题到最终的定稿,老师都给予了耐心的指导,帮助我解决
撰写过程中的诸多难题。
三年来,在孔新兵老师的指导下我对所学专业有了更深层次的认识,为学为人方面都有了很大进步。
临近毕业,我已有了一定的知识储备和实习经验,即将走上工作岗位,老师的谆谆教诲,
我会铭记在心。在今后漫长的工作生涯中,我还会不断向老师继续请教和学习。

另外,感谢南京审计大学和统计与数学学院给我的培养机会。我本科并不是相关专业,
后由于各种机遇来到南审,进行应用经济学统计学方向的学习和研究,这对我的知识面是极大的
拓宽,这三年的学习经历对我今后的终身学习具有莫大的帮助。
毕业后我将从事软件开发工作,随着岁月的流逝,我可能会忘记在学校学习过的一些课程,可能会
记不起一些教授过我课程的老师。但是在这三年中,学校和学院老师们脚踏实地的
态度和积极乐观的性格将默默影响我今后的成长。

另外,感谢和我朝夕相处的研究生同学们。感谢他们在我专业基础薄弱时给予的耐心帮助以及
一起参加学科竞赛时大家的合作无间。三年的时间很快就过去了,
我的性格也变得更加成熟,认识了越来越多的朋友。每一位同学身上都有太多值得我学习的地方,
毕业只是我们学生生涯的结束,与各位同学的宝贵友谊将会伴随我们终身。

最后,感谢学校和学院给我们安排评阅论文的各位专家、教授,感谢各位提出的宝贵意见,
使得我的论文更加完善。
\vspace{2ex}

\rightline{ 蒯强\ \ \ \ \ }
\vspace{1ex}
\rightline{2021年5月 }