\section{近似因子模型的$L_1$范数主成分估计算法研究}

本章继续讨论$L_1$范数主成分分析,重点在于优化前文介绍的交替凸优化算法。
$L_1$主成分分析虽然在稳健性上有很大优势,但是其求解需要比$L_2$主成分分析
多得多的时间。

在第\ref{chapter3}中的研究中,已经重点讨论了优化最小绝对值回归的算法,
注意到交替凸优化算法中包含了最小绝对值回归问题,
因此可以充分利用得出的结论,尝试将两种算法应用于改良交替凸优化算法。
之后通过一个模拟实验和对一个高频金融数据的主成分分析的测试,
测试了算法的性能。

至此我们已经对问题$P_3$做出了较为深入的讨论,而问题$P_4$同样值得探究,它是$L_1$主成分分析的另一重要研究方向。
我们将介绍一种利用贪婪策略求解$P_4$算法,并通过模拟实验证明算法的有效性。

在第\ref{chapter2}章,我们将交替凸优化算法用于近似因子模型的估计,并且通过实证研究论证了
$L_1$范数主成分分析得出的因子同样适用于宏观经济预测中,并且有良好的预测表现。
本章的最后,我们将求解$P_4$来得到的因子估计量同样在国内主要月度宏观经济数据预测场景下进行了测试,
并给出了一些结论与建议。

\subsection{改进的$L_1$主成分分析交替凸优化算法}
在第\ref{chapter2}章中,我们提出了求解$L_1$主成分分析的交替凸优化算法;
交替凸优化算法的计算量主要集中在\eqref{subpro}和\eqref{subproabs},注意到
在数据服从因子模型的假设下,\eqref{subproabs}是一个最小绝对值回归问题。
因此我们可以尝试使用第\ref{chapter3}章中基于替代变量的估计算法来代替线性规划求解\eqref{subproabs}。

\subsubsection{基于替代变量算法的交替凸优化算法改进}
假设
\subsubsection{基于聚类——迭代拆解算法的交替凸优化算法改进}

\subsubsection{数值模拟}

\subsubsection{高频金融数据的主成分分析}

\subsection{求解特征空间$L_1$范数最大化的贪婪算法}

我们已经在第三章中研究了$P_3$,这里我们首选给出问题$P_4$的规范化表述:
\begin{equation}\label{p4}
    P_4: \ \hat{\bm A} = \underset{\bm{A}}{\operatorname{arg \ max}} \| \bm A^T \bm X\|_{L_1}
    = \underset{\bm A}{\operatorname{arg\ max}} 
    \sum_{i=1}^{n}\sum_{k=1}^{m}|\sum_{j=1}^p{a}_{jk}x_{ij}|
     \text{,其中}\bm A^T\bm A = \bm I_m
\end{equation}

这里我们介绍一种求解$P_4$的算法。求解\ref{p4}时,
由于对于不同的$m$,最优$\bm a_j^*$会发生变化,因此想要得出\ref{p4}的全局最优解是非常复杂的。
下面介绍Kwak于2009年提出的一种算法,它将问题分解一个个$m=1$的子问题,然后用贪婪策略获得全局解。
该算法直观、简洁并且易于实现。
\subsubsection{求解子问题}

首先考虑$m = 1$的情况,该问题寻找\eqref{m-1}的最优解。
\begin{equation}\label{m-1}
    \bm a^* = \underset{\bm a}{\operatorname{arg\ max}}\| \bm a ^T \bm{X}\|_{L_1}
    = \underset{\bm a}{\operatorname{arg\ max}}\sum_{i=1}^n|\bm a^T\bm x_i|
    \text{,其中}\|\bm a\|_{L_2} = 1
\end{equation}

算法4.1给出了\eqref{m-1}的求解算法。
\begin{table}[H]%%%%%%开始表格
    \centering%把表居中
    \begin{tabular}{{p{0.9\columnwidth}}}%三个c代表该表一共三列,内容全部居中
    
    \toprule%第一道横线 表头
    算法4.1: $m=1$时的$P_4$求解算法\\
    \midrule%第二道横线 符号+解释+单位 中间用&隔开
        初始化:取$\bm a$的任意初始值;$\bm a^0\leftarrow \bm a^0/\|a^0\|$。 \\
        对于$t = 1, ..., T$:\\
            1.对所有的$i \in \{1, ..., n\}$,若$\bm a^{0T}\bm x_i < 0$,令
            $p_i^t = -1 $,否则$p_i^t = 1$ 。\\
            2.令$t \leftarrow t+1$并且$\bm a^t = \sum_{i=1}^np_i^{t-1}\bm x_i$;
            $\bm a^t \leftarrow \bm a^t/ \|\bm a^t\|_{L_2}$ 。\\
            3.检查是否收敛:\\
            1)如果$\bm a^t \neq \bm a^{t-1}$,回到步骤2;\\
            2)若存在某个$i$使得$\bm a^T\bm x_i = 0$,令$\bm a^t \leftarrow
            (\bm a^t + \Delta \bm a)/\|\bm a^t + \Delta \bm a\|_{L_2}$并回到步骤2,这里
            $\Delta \bm a$为一个很小的随机向量; \\
            3)若前面两个都不符合,令$\bm a^* = \bm a^t$,算法返回。\\
    \bottomrule%第三道横线
    \end{tabular}
\end{table}%%%%%%结束表格
按照算法4.1得出的$\bm a^*$是$\bm x_1, ..., \bm x_i$的线性组合,即$\bm a^t \propto \sum_{i=1}^np_i^{t-1}\bm x_i$,
故其拥有旋转不变性。注意到的$a^*$为\eqref{m-1}的最优解,它可能不是\eqref{p4}的全局最优解。
$\bm a^0$可以任意设置,因此可以采用$L_2$主成分作为其初始值,从而提高解趋向全局最优的可能性。

\subsubsection{贪婪策略求解主成分}
前文已经给出了获得$m = 1$时$P_4$最优解的算法,我们可以通过贪婪策略来求解$P_4$。贪婪策略是一种
在问题的动态求解过程中,对问题所处的每一个当前状态都寻求其最优解,从而期望能够逼近全局最优解的算法设计思想。
下面给出利用贪婪策略求解$P_4$的步骤,
\begin{table}[H]%%%%%%开始表格
    \centering%把表居中
    \begin{tabular}{{p{0.9\columnwidth}}}%三个c代表该表一共三列,内容全部居中
    
    \toprule%第一道横线 表头
    算法4.2: 求解特征空间$L_1$范数最大化的贪婪算法\\
    \midrule%第二道横线 符号+解释+单位 中间用&隔开
        输入:$\bm a^0$,$\{\bm x_i^0 = \bm x_i\}_{i=1}^n$。\\
        对于$j = 1, ..., m$: \\
        对所有的$i \in \{ 1, ..., n\}$,$\bm x_i^j = x_i^{j-1} - \bm a^{j-1}(\bm {a^{(j-1)T}}\bm x_i^{j-1})$ , \\
        通过算法4.1求解$\bm a^j$,令$\bm{X}^j = [\bm x_1^j, \bm x_2^j, ..., \bm x_i^j]$。 \\
    \bottomrule%第三道横线
    \end{tabular}
\end{table}%%%%%%结束表格

\subsubsection{模拟与实证}

\subsection{本章小结}
本章进一步讨论了$L_1$范数主成分分析的求解算法问题,
介绍了$L_1$范数主成分分析的另一种求解思路。