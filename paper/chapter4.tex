\section{$L_1$范数主成分分析算法研究}
本章继续讨论$L_1$范数主成分分析,重点在于讨论问题$P_3$和$P_4$的求解方法。
首先本文介绍目前适用于$L_1$范数主成分分析的几种主要求解算法,
阐述算法的基本思想,以及其主要优缺点。

问题$P_3$和$P_4$不是一组对偶问题,求解它们可以得到不同的$L_1$主成分,
$P_4$问题的解也具有一些不同的性质,在实践中也已经得到了应用。
因此本章重点讨论了求解$P_4$的一种典型算法。

对于$P_3$,我们在第三章中已经阐述了
求解$P_3$的交替凸优化算法。本章将结合第二章中介绍的两种加速算法,
我们提出了一种结合两种算法优点的快速方法,该方法可在一定程度上
提升交替凸优化算法的性能。

我们通过一个带噪声的人脸图像重构实验观
察两种算法的性能表现,并结合理论和实际应用分析算法的优缺点;

在第\ref{chapter3}章,我们将$L_1$范数主成分分析用于近似因子模型的估计,并且通过实证研究论证了
$L_1$范数主成分分析得出的因子同样适用于宏观经济预测中,并且有良好的预测表现。
本章的最后,我们将求解$P_4$得到的因子估计量也在国内主要月度宏观经济数据预测场景下进行了应用,
并给出了一些结论与建议。

\subsection{问题和理论介绍}
我们已经在第三章中研究了$P_3$,这里我们首选给出问题$P_4$的规范化表述:
\begin{equation}
\end{equation}

\subsection{一种实现旋转不变性的$L_1$范数主成分分析算法}

\subsection{改进的$L_1$主成分分析交替凸优化算法}

在第三章中,我们提出了

\subsubsection{SVN-AID算法}
在第二章中我们已经研究了AID和SVN两种方法,注意到AID算法是一种具有普适性的算法思想。我们不妨再次分析使用AID算法
解决最小绝对值回归问题的实质:为了避免在整个数据集上一次性求解,我们使用聚类的方法,希望通过尽可能少的
计算来利用样本的信息。不妨将前文提到的AID算法称为AID-LP算法,它成功减少了解决最小绝对值回归问题的LP计算规模。

在AID算法求解最小绝对值回归时,
假设在第$t$次迭代中我们得到了$\bm{\beta}^t$,然后根据$\bm{\beta}^t$对现有的聚类进行拆解。
在下一步,我们重新计算该问题,直到达到问题最优解。而$\bm{\beta}^t$此时已经成功分离了许多聚类,
而这些聚类在下一个步骤中仍然全部需要进入计算,因为上一步的参数信息$\bm{\beta}^t$在下一步计算时被舍弃了。

而SVN算法,正是通过利用上一步的参数信息来构造代理变量简化计算。我们不妨结合两种算法的优点,将SVN算法
和AID算法结合起来,进一步优化对最小绝对值回归问题的求解。

\subsubsection{算法步骤}
以下我们给出SVN-AID的步骤:
1)初始化:随机采样,进行最小绝对值回归,该步骤使用LP,得到$\hat{\bm{\beta}}^0$。根据各个样本点对该模型的拟合残差进行聚类。
2)记上一次迭代得出的系数估计为$\hat{\bm{\beta}}^t$,对当前聚类按照规则拆解。本步骤返回被拆解的聚类组成的集合,
舍弃上一步正确分类的聚类。
3)根据上一步返回的聚类集合,构造新数据集,并且按照\eqref{svny}构造代理变量数据集,并由\eqref{yl2loss}得出本轮估计。

具体见算法4.2,该算法通过舍弃正确分类的聚类,不断缩小问题规模,迭代优化上一步得出的参数估计值来进行对最优解的近似,
该算法在理论上可以极大减小AID算法的计算量,并且在舍弃大量样本点的同时,充分利用了前面步骤得出的参数信息。

\begin{table}[H]%%%%%%开始表格
    \centering%把表居中
    \begin{tabular}{{p{0.9\columnwidth}}}%三个c代表该表一共三列,内容全部居中
    \toprule%第一道横线 表头
    算法4.2 SVN-AID算法\\
    \midrule%第二道横线 符号+解释+单位 中间用&隔开
    输入:$Y$和$\bm{X}$的样本$\bm{Y} = (Y_1, Y2, ..., Y_n)$,$\bm{X} = (\bm{X}^T_1, \bm{X}^T_2, ..., \bm{X}^T_n)$,
    迭代次数$T$,核函数$K$,依赖于迭代次数的带宽$h_t(t = 1, ..., T)。$
    \\
    初始化:给出初始估计,$\hat{\bm{\beta}}^{0} $,对每一个样本点$(X_i, Y_i)$根据$Y_i - X_i^T\bm{\beta} $的结果进行
    K-means聚类,得到聚类结果$C^0 = \{C_1^0, C_2^0, ... , C_K^0\}$。
    \\
    对于$t = 1, ..., T$:\\
        1)根据上一轮$\hat{\bm{\beta}}^{t-1}$,计算带宽$h^t$和
        $\hat{f(0)}^t = \frac{1}{K^th^t}\sum_{i=1}^{K^t}K(Y_i - \bm{X}_i^T\hat{\bm{\beta}}^{t-1})$。\\
        据\eqref{svny}在当前聚类上构造替代变量$\tilde{Y} = (\tilde{Y}_1, ..., \tilde{Y}_{K^t})$;
        求解
        $$
            \hat{\bm{\beta}}_t = \underset{\bm{\beta} \in \mathbb{R}^{p}}{\operatorname{arg\ min}}
            \frac1{K^t} \sum_{k=1}^{K^t}|C_i^{t}|(\tilde{Y}_k - \bm{X}_k^T\bm{\beta})^2
        $$
        2)对于$k = 1, ..., K^t$,计算$\theta_i = y_i - \sum_{j \in J}x_{ij}\hat{\bm{\beta}}_j^t$,
        根据$\theta_i$符号异同,将$C^t_k$分成两个集合,$C_{k+}^t = \{i \in C_k^t | \theta_i > 0\}$ ,
        $C_{k-}^t = \{i \in C_k^t | \theta_i < 0\}$,这两个集合在下一步形成新的聚类,
        即$C^{t+1} \leftarrow C^{t+1}\bigcup \{ C_{k+}^t, C_{k-}^t\}$。 
        
        输出:$\bm{\beta}^{T}$
    \\
    \bottomrule%第三道横线
    \end{tabular}
\end{table}%%%%%%结束表格

\subsubsection{数值模拟实验}

\subsection{人脸图像去噪实验}

\subsection{基于国内主要月度宏观经济数据的实证研究}

\subsection{本章小结}