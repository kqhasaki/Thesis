\section{绪论}
本章阐述了本篇论文的研究背景和研究意义,综述了国内外学者在相关领域的研究现状、主要的成果、结论和观点。
由此展开,叙述了论文的主要研究内容、研究所使用的技术路线、本文的主要创新点和行文的结构。

\subsection{研究背景及意义}
在宏观经济学理论中,政府是指导和调控经济运行的主体。国民经济的发展和政府的指导和调控紧密相关,政府部门需要时刻把握国民经济的方方面面,
不仅仅要关心主要产业GDP的增长,还要关心经济中的通货膨胀、利率和汇率等等许多复杂的经济变量,并且需要研究这些经济变量发生变化的原因
以及各种经济变量之间的作用关系。只有这样,政府才能发现经济中存在的问题,并且给出针对性的指导和调控手段。

在古典经济学理论中,对于国民经济作为一个整体的活动情况一般不予考察,古典经济学往往从围观主体出发来解释观察到的宏观经济问题。然而,在市场经济
高度发展的当代,市场规模空前扩大,对人类社会已经产生了深远影响。宏观经济的发展已经密切关系到民生发展,宏观经济的动荡也通过金融危机和失业
直接关系到社会稳定。特别是一战后的大萧条使得古典经济学对宏观经济的分析陷入了困境,在凯恩斯理论的影响下,现代宏观经济学进入主流。
在现代宏观经济学的主张下,政府应该采取随机应变的宏观经济政策,通过财政和货币手段对国民经济发展进行调控和指导。

政府管理国民经济的关键是制定正确的经济政策,而做决策的前提就是预测。因此制定经济政策不仅需要各级政府对经济发展的当前状态有充分把握,
并且能够对经济的未来变化有一定的预测能力。经济预测是根据经济发展的时间和空间纬度的综合信息来试图研究经济变化趋势的技术。政府为了提高
经济政策的有效性,就需要通过更加准确的预测来分析和判断信息,因此政府部门需要不断追求更加准确的经济预测技术,才能在复杂的经济形式变化中
掌握主动权。

除了经济发展的需要,为了规避国民经济的风险,把控我国宏观经济安全大局,也对政府更加准确地把握和预测国民经济提出了要求。
当前我国正处于经济和社会向前发展的关键时期,经济虽然多年持续高速发展,但是,经济结构发展不均衡的现象逐渐显现,
社会经济系统的复杂性也日渐增强。这就更需要对经济系统进行实时准确的检测和对未来经济形势准确预测,提前做好准备工作,
防范经济增长的重大风险。经济风险不仅仅和宏观经济运行状况紧密关联,同样和国计民生密不可分,
不管是通过对经济系统的宏观把控还是对经济运行中某些指标的局部观察,我们都可以了解经济运行的一些趋势,
并且通过对经济中的某些重要运行趋势进行预测,从而规避经济风险,稳定宏观经济安全大局。
把控宏观经济安全大局对于国民经济平稳快速发展、改善国计民生、促进经济又好又快发展、促进国家安全稳定有重要的战略和实际意义。
习近平总书记于2014年首次提出总体国家安全观,并构建了集政治安全、国土安全、军事安全、经济安全和文化安全等领域为一体的国家安全体系。
其中,经济安全是国家安全体系的重要组成部分,是国家安全的基础。宏观经济安全问题需研究1个或多个指标(如经济增长量)与其他指标(如资本、劳动力、人口和技术等)相互间的关系,
从而揭示经济现象背后的经济规律,发现经济运行风险点,制定经济政策。为了避免宏观经济中的冲击带来的风险,
需要对宏观经济总量进行实时预测和短期预测来减少不确定性,为宏观经济调控政策提供决策支持。通过对宏观经济指标进行有效监测和预测,
并提前识别和防范相关风险,对保证国民经济稳步增长具有重要意义。

宏观经济预测方法繁多,常见的方法是使用经济学方程组来给宏观经济系统建立模型\cite{郭崇慧2001宏观经济预测模型体系研究},
从而通过模型来对某些宏观经济指标进行预测并且给出经济学分析。
然而国民经济是一个非常复杂的整体,它是一个内部极其复杂的系统,并且经常伴随难以预测的外部冲击,对于复杂的经济变量,要做到准确预测往往非常艰难。
虽然现在传统的经济学理论和现代计算机技术的结合下,已经发展出了许多预测方法,并且学者们仍在不断利用新的理论和技术来完善这些已有的方法。然而,
在我国国情下,进行准确的经济预测仍然非常困难。现代经济系统极为复杂,
意味着使用这种方法我们就可能需要建立非常复杂的经济分析模型,有可能导致分析困难,并且往往预测结果和真实的经济运行结果有较大出入。
因为系统过于复杂,学者们就开始使用基于历史数据进行向前预测的经验方法(Sargent, 1997\cite{sargent1977business};
Stock and Watson, 1989\cite{stock1989new},1993\cite{stock1993simple},1999\cite{stock1999forecasting}\cite{stock1999business}),
这种方法大量应用数据和数据分析、大数据技术和统计学知识,
甚至应用了一些人工智能领域的方法\cite{王维2000人工神经网络在非线性经济预测中的应用},
并且融合了成熟的经济学理论和模型\cite{白仲林2014两类},往往能够起到不错的效果。在现代经济系统中,经济社会中每时每刻都在产生大量的数据,
这些海量的经济活动数据给经济预测提供了丰富的数据支撑,如何有效地利用大数据时代丰富的数据资源进行宏观经济现时预测已成为一个重要的研究课题。
当今的宏观经济数据具有高维度、数量多、处理困难等特征。结合计算机科学和统计学等方面的技术和理论知识,
我们能够采取一些方法来处理数据,从中挖掘提炼出有效的信息。关于这个方面的研究正在火热进行,人们不断尝试使用更新的技术来处理数据,
提取信息,希望利用高维度下丰富多面的经济信息来对未来的经济发展趋势作较为准确的分析。

由于宏观经济变量往往具有维数很高、分布重尾、数据集不完整等特征\cite{qiu2015robust},导致在宏观经济的数量研究中遇到稳健性问题。
例如,在线性回归模型的估计、主成分分析、因子模型的估计等场景下,基于最小二乘法的经典方法往往难以得到满意的结果。
因此,在宏观经济数据的分析和预测中往往还需要引入更加稳健的方法,例如使用最小绝对值回归代替最小二乘回归即是一种
常见的做法。

基于$L_1$范数的稳健方法在宏观经济预测中已经得到了一定的应用,例如最小绝对值回归等。
高维宏观经济数据可以通过因子模型来分析和处理,得到一些对于经济预测很有价值的信息,
然而由于经济受到外部冲击等原因,许多宏观经济指标不能够服从正态分布,往往在分布上呈现重尾特征,
这就使得普通主成分分析方法在估计因子模型时产生偏差。因此,稳健因子分析方法的研究成为迫切的需求,
也成为当前研究的热点问题。

\subsection{基本概念和研究现状综述}
本文的研究内容是基于$L_1$范数主成分分析的因子模型统计推断方法及其在宏观经济中的应用。
本节主要围绕近似因子模型和$L_1$范数主成分分析两大主题进行了文献的搜集和整理。
下面简要介绍它们的基本概念,并梳理了关于这两个主题国内外学者在相关领域的研究现状和存在问题。

\subsubsection{因子模型及其在宏观经济中的应用}
因子模型最早由英国心理学家Charles Spearman提出,最早用于智力测验得分数据的分析中。
提出因子模型是为了使用少量潜在的、不可观测的因子来描述多个相依变量间的关系。
假设一组变量能够用因子模型来描述,那么我们就把这组变量称为观测变量,因为它们的值可以实际测度,
而观测变量的值是由一组不可观测的因子的线性组合决定的。

经典因子模型提出后,迅速在心理学、社会学等多个学科内得到了广泛的使用。
然而因子模型在宏观经济研究领域的应用出现较晚,这可能是因为经典因子模型主要用于处理截面数据,
而宏观经济研究中主要研究的数据为时间序列数据。
然而近年来因子模型的理论和应用已经得到了很大的完善和发展,
作为一种重要分析工具因子模型已经成为计量经济学的必修内容,并且动态因子模型
常被应用于宏观经济政策的评估、经济指数的构建和经济指标的预测中。

因子模型在宏观经济研究中的广泛使用和当今宏观经济数据集的特征密切相关。
从政府部门和一些经济研究机构公布的数据和报告可以发现,在制定宏观经济政策时,政策制定者
在决策过程中考虑到了非常多的经济变量。一个简单的经济政策背后可能需要分析成百上千不同的经济指标,
并且这些经济指标的可靠数据记录往往较少,许多宏观经济指标往往只有30年以内的可靠记录。
因此宏观经济研究中面对的数据较为特殊,常用的计量经济模型例如向量自回归模型等在这种数据集下
很难可靠地估计模型系数。在建立模型时将所有的经济指标都进行研究,大大提升了问题的计算复杂度,
难以快速给出分析结果。而因子模型可从高维数据集中提炼出少量公共因子,
这些因子中包含了经济运行的许多有效信息,因而可以通过分析这些公共因子准确
了解宏观经济的运行状况。

经典因子模型适用于截面数据,对于分析经济时间序列而言不十分适用,因此国内外学者们在经典因子模型上发展出了动态因子模型。
Geweke\cite{geweke1977dynamic}与Sargent\cite{sargent1977business}等人于1977年分别提出了动态因子模型,
动态因子模型是经典因子模型在时间序列数据上的延伸,它提供了从维数众多的经济时间序列数据中提取公共因子来研究和解释
经济波动的手段。
一般认为动态因子模型的提出主要基于以下观点:宏观经济具有周期性的波动是通过一系列经济变量的活动来传递和扩散的,
任何一个经济变量本身的波动都不足以代表宏观经济的整体波动。对于经济波动的研究需要从多个具有相依性的经济变量同时着手。
因此,需要从一国许多经济时间序列数据中估计出驱动各变量波动的共同动态因子,并且给出解释。
动态因子模型已经成为判别和分析经济周期波动的有效工具。

动态因子模型主要应用于仅包含几个总体宏观经济变量的小型数据集,这主要是由于经济学家们假定影响所有变量的普遍性结构
冲击是经济变量协同变动的原因。但是在包含了成百上千个变量的高维宏观经济数据集中,常常存在某个仅影响某一组变量的部门冲击或是局部冲击,
若根据经济理 论将这些非普遍性冲击归于异质性部分,那么就可能造成异质性部分的截面相关,这又将违背经典因子模型中对于异质性部分相互正交的假定。
Chamberlain和Rothschild于1983年提出应该放松关于异质性部分协方差矩阵为对角阵的假定,即允许其为非对角阵,
从而发展出了近似动态因子模型\cite{chamberlain1982arbitrage}。在众多因子模型及其拓展模型中,近似动态因子模型得到了
最为广泛的使用。

Forni等人于2000年提出应该在近似动态因子模型的基础上,进一步放松假设,即允许观测变量受到因子的无限滞后项的影响,从而
发展出了广义动态因子模型\cite{forni2000generalized}。但是在模型的参数估计方面,广义动态因子模型只能使用频域方法估计,
因此其使用范围比较有限。

因子模型在宏观经济分析中有两个最主要的应用:1)构建指数。最早的尝试是Stock和Watson等于1989年提出的一致和先行景气指数
\cite{stock1989new}。另一个例子是根据Stock和Watson在1999年提出的方法\cite{chan1999dynamic},
基于美国的85个月度宏观经济数据构建了著名的CFNAI指数。
2)预测。Stock和Waston于2002年提出了扩散指数模型\cite{stock2002macroeconomic},使用因子的主成分估计量进行预测。
通过少量的因子,不仅利用了大量的信息而且避免了变量过多带来的的模型自由度问题。使用美国宏观经济数据,Eickmeier等
于2008年进行的实证研究\cite{eickmeier2008successful}证明了使用扩散指数模型的预测效果要好于大部分其他方法。

目前在国外已经存在大量关于使用因子模型的进行宏观经济预测的实证研究(详见综述文章\cite{luciani2014large})。
国内已有学者应用动态因子模型应用于宏观经济分析与预测中。例如杜勇宏等于2011年基于国内宏观经济数据进行了GDP预测,
实证结果表明动态因子模型的预测效果好于ARMA模型\cite{杜勇宏2011动态因子模型与};杜海韬等于2013年采用结构动态因子方法
对经济冲击进行了分析\cite{杜海韬2013部门价格动态};部分学者进行了利用因子模型构建中国金融景气指数\cite{韩艾2010广义动态因子模型在景气指数构建中的应用}
的尝试,并且将景气指数等应用于经济预测中\cite{何问陶2007我国宏观经济先行指标体系及对经济预测实证研究};刘汉等于2011年
基于混频数据对中国宏观经济总量的实时预测问题进行了研究\cite{刘汉2011中国宏观经济总量的实时预报与短期预测}。
但总体而言,目前在国内使用因子模型进行宏观预测的实证研究仍
较少(详见综述文章\cite{高华川2015动态因子模型及其应用研究综述})。

\subsubsection{$L_1$范数主成分分析}
主成分分析是一种应用广泛的数据降维技术。通过选择高维数据在特征空间中的数个基向量构成投影,
这些基向量代表了高维数据在特征空间的投影方差最大的几个方向。对应到优化理论,传统主成分分析算法解决
$L_2$范数优化问题,因而也称为$L_2$范数主成分分析。

而面对噪声干扰高的数据,$L_2$范数主成分分析
不具有稳健性。而在优化理论中,$L_1$范数的优化问题对噪声不敏感,具有很好的稳健性。
若将$L_1$范数应用于主成分分析中,这种方法就称为$L_1$主成分分析,在理论上它具有很好的稳健性。
因此,国内外学者从很早就开始了对$L_1$范数主成分分析的研究。
众多应用性的研究(Galpin and Hawkins, 1987\cite{galpin1987methods};
Baccini et al.,1996\cite{baccini1996l1};Ke and Kanade, 2003\cite{ke2003robust};
Ding et al., 2006\cite{ding2006r};Choulakian, 2006\cite{choulakian2006l1};Gao, 2008\cite{gao2008robust}
;Kwak, 2008\cite{kwak2008principal};Jeon et al, 2018\cite{jeon2018data})
已经表明,$L_1$范数主成分分析具有很强的抗噪声能力,应用于机器视觉\cite{ke2005robust}、
人脸识别\cite{kwak2008principal}等领域有非常好的效果。

另外还可以采用$L_1$范数主成分分析估计中位数因子模型。中位数因子模型是分位数因子模型的一个特例,
在分位数因子模型中我们对观测变量和潜在因子的分位数进行建模,研究表明该模型在处理重尾数据时更加稳健(Chen L et al, 2018\cite{chen2018quantile};
Bunn D et al, 2016\cite{bunn2016analysis};Gonccalves et al, 2020\cite{gonccalves2020bayesian};Ma et al, 2020\cite{ma2020estimation})。

然而$L_1$范数主成分分析在求解上比$L_2$范数主成分分析困难许多,
具体体现在难以获得全局最优解、计算复杂度高等方面。
$L_1$主成分分析有两种问题形式,一种是最小化重构误差的$L_1$范数;另一种是最大化特征空间投影的$L_1$的范数。

对于前者,
一个具有代表性的早期研究\cite{baccini1996l1}提出了一种基于典型相关分析的算法,该算法具有一定的启发意义;
Ke和Kanade等人于2005年提出了一种方法\cite{ke2005robust},该算法将全局的非凸优化问题使用
一种交替的凸优化算法解决,每次交替求解的问题可以获得全局最优,每一次交替中需要求解若干个线性规划问题;
Gao于2008年提出一种在Laplace噪声下估计最优$L_1$特征空间的Bayesian方法\cite{gao2008robust};
而Brooks等人于2013年提出一种算法
\cite{brooks2013pure},该算法逐次降低优化问题的维数,递进地寻找从$m$维特征空间向$m-1$维特征空间的最佳投射,而获得这样的最佳投射
需要求解一系列最小绝对值回归问题;Park等人于2016年提出一种算法\cite{park2016iteratively},在迭代中使用一种再加权最小二乘法求问题的近似解。
而对于该问题,至今没有一种算法能够给出全局最优解。

对于后者,Choulakian于2006年对该问题进行了完整描述\cite{choulakian2006l1},基于协方差矩阵的$L_1$估计\cite{galpin1987methods}来求解;
比较具有代表性的是Kwak于2008年提出的一种贪婪算法\cite{kwak2008principal},该算法在给出前$k$个主成分的局部最优估计下,
求解第$k+1$个主成分的最优解;Nie等人于2011年扩展了该算法,并给出一种采用非贪婪策略的实现方式\cite{nie2011robust};
Markopoulos等人于2014年给出了一种在特征维数已知情况下可获得全局最优解的算法\cite{markopoulos2014optimal},并且
证明了在原始数据维数和样本数一起趋于无穷时,本问题是一个NP难题,另外该结论同样被Mccoy\cite{mccoy2010two}等人证明;
Park等人于2020年提出了一种基于聚类——迭代拆解的解法\cite{park2021optimization},在没有进行理论论证的情况下进行了数值模拟,
实验结果表明了该算法在性能上有所改善。

\subsection{研究创新}

1)近似因子模型在宏观经济研究中被广泛使用,其因子的主成分估计量在宏观经济预测中能够起到很好的效果。
然而,传统主成分分析本质上求解是$L_2$范数优化问题,面对高维重尾宏观经济变量在数据处理上不具有稳健性。
$L_1$范数主成分分析作为$L_2$主成分分析的一种稳健替代手段,在信号处理、机器视觉和人脸识别中
已经有了大量应用。
本文创新性地采用$L_1$范数主成分分析作为近似因子模型的估计方法,来代替$L_2$主成分分析。
采用$L_1$范数主成分分析估计得到的因子和$L_2$是不同的,本文通过基于国内高维月度宏观经济数据集的实证研究,
比较了两种不同因子在宏观经济预测中的表现,论证了将$L_1$主成分分析作为近似因子模型估计方法来进行宏观
经济预测的稳健性和统计精确性。

2)$L_1$范数在宏观经济预测中的一大应用是采用最小绝对值回归建立线性模型,而该模型
求解采用的线性规划方法在当今数据维数越来越高、数据量越来越大的场合下计算效率较低,因而限制了
最小绝对值回归在这一领域的应用。本文阐述了两种适用于最小绝对值回归性能优化的最新算法,
它们在应对大规模和高维数据集上较传统线性规划方法有着更好的性能表现,
本文通过数值模拟实验测试和分析了两种算法的优缺点,并且给出其在实际应用中的一些建议。
并在两种算法的基础上,融合彼此的优点提出了一种新的算法,通过模拟实验说明,该算法
在一定程度上同样可以提升最小绝对值回归的计算性能。

3)本文对$L_1$范数主成分分析的算法根据目标问题不同,可以有两种不同的解法。
本文分别介绍了一种求解重构误差最小化的交替凸优化算法和一种求解特征空间$L_1$范数最大化的算法,
在国内月度宏观经济数据集上使用两种算法分别进行了近似因子估计,结果表明两者同样适用,并给出一些应用建议。
最后,本文结合对最小绝对值回归性能的研究,改进了交替凸优化算法,使其在估计高维近似因子模型时拥有更好的性能。

\subsection{本文结构}

本文主要内容分为5个章节:

第一章为绪论。该章节介绍了本文的写作背景和课题的研究意义,并且分析了课题相关的理论和应用研究
的进展和不足,指出了本文写作的创新点。

第二章为基于$L_1$范数主成分分析的高维因子模型与估计方法。该章节介绍了因子模型的基本概念,
提出了将$L_1$范数主成分分析引入近似因子模型的估计,介绍了一种求解$L_1$范数主成分分析
的交替凸优化算法并给出了具体步骤。
接下来,通过模拟实验探究$L_1$范数主成分分析相比于
传统基于$L_2$范数的主成分分析的稳健性。
最后,通过一个基于国内月度高维宏观经济数据
的实证研究,来验证$L_1$主成分分析能否应用在宏观经济预测中。

第三章为最小绝对值回归的性能研究。
该章节介绍了最小绝对值回归,它是稳健$L_1$范数在宏观经济建模中的另一个主要应用。
我们说明了最小绝对值回归的稳健性,介绍了其基本求解方法,并针对其求解性能问题进行了研究。
这里介绍了两种适用于高维最小绝对值回归的最新算法:聚类——迭代拆解算法和
基于替代变量的估计方法,详细阐述了两种算法的基本思想和具体步骤。
通过模拟高维宏观经济数据的数值实验,测试了算法的优缺点,并给出建议。
本章的研究有助于我们改进$L_1$主成分分析的交替凸优化算法的求解性能。

第四章为$L_1$范数主成分分析求解算法。
第二章已经将$L_1$范数主成分分析引入了近似因子模型的估计,并探究其应用于经济预测中的可行性。
这一章将进一步讨论$L_1$范数主成分分析,较深入地研究其求解问题。
这里首先利用第三章的研究成果
对交替凸优化算法的求解性能做出了优化。我们进行了数值模拟实验,
来测试算法的表现。
并且我们介绍了一种解决低维投影$L_1$范数最大化问题的求解算法,它和前面的交替凸优化算法
求解$L_1$主成分分析的角度有所不同,
我们同样进行了基于国内月度宏观经济数据的实证研究,探究该方法是否同样适用于宏观经济预测中,
并对实际应用中$L_1$算法的选择提出了建议。

第五章为结论与展望。
该章节总结了本文的主要研究结论和存在的问题,并为进一步的研究提出了方向。
