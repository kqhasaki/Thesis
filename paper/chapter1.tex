\section{绪论}
本章阐述了本篇论文的研究背景和研究意义,综述了国内外学者在相关领域的研究现状、主要的成果、结论和观点。
由此展开,叙述了论文的主要研究内容、研究所使用的技术路线、本文的主要创新点和行文的结构。
\subsection{研究背景及意义}
在宏观经济学理论中,政府是指导和调控经济运行的主体。国民经济的发展和政府的指导和调控紧密相关,政府部门需要时刻把握国民经济的方方面面,
不仅仅要关心主要产业GDP的增长,还要关心经济中的通货膨胀、利率和汇率等等许多复杂的经济变量,并且需要研究这些经济变量发生变化的原因
以及各种经济变量之间的作用关系。只有这样,政府才能发现经济中存在的问题,并且给出针对性的指导和调控手段。

在古典经济学理论中,对于国民经济作为一个整体的活动情况一般不予考察,古典经济学往往从围观主体出发来解释观察到的宏观经济问题。然而,在市场经济
高度发展的当代,市场规模空前扩大,对人类社会已经产生了深远影响。宏观经济的发展已经密切关系到民生发展,宏观经济的动荡也通过金融危机和失业
直接关系到社会稳定。特别是一战后的大萧条使得古典经济学对宏观经济的分析进入了困境,在凯恩斯理论的影响下,现代宏观经济学进入主流。
在现代宏观经济学的主张下,政府应该采取随机应变的宏观经济政策,通过财政和货币手段对国民经济发展进行调控和指导。

我国目前仍然是发展中国家,并且经过改革开放的40年,经济已经得到了巨大的发展,我国的国民经济仍然处在一个高速发展的上升阶段。在党的指导下,
政府制定了保持国民经济又好又快发展的目标,意味着政府部门需要通过政策手段来保持国民经济长期平稳发展。并且,在我国,中央和地方政府享有一定的
财政收入,并且能够制定各种不同的收费和税收政策来影响投资、消费和公共事业等。因此我国政府具有很好的条件来采取随机应变的宏观经济政策。
我国在改革开放以来,一直保持着惊人的经济增长速度,国家和人民也从经济发展中取得了巨大的红利。然而,发展中也伴随着许多风险和挑战,我国经济
受到产能问题、市场化不充分、过度依赖财政政策等问题的困扰,加上金融危机、贸易纠纷等外部冲击的作用,自从2012年以来,我国经济已经进入了
发展的新常态阶段,我国经济面临很大的下行压力,在以习近平总书记为核心的领导下,我国经济发展需要实现新的目标:减少城乡差距、
优化经济产业结构和构建创新驱动型经济,为此,各级政府正在充分发挥宏观手段,试图化解经济中的风险和为经济增长注入新的活力。

政府管理国民经济的关键是制定正确的经济政策,而做决策的前提就是预测。因此制定经济政策不仅需要各级政府对经济发展的当前状态有充分把握,
并且能够对经济的未来变化有一定的预测能力。经济预测是根据经济发展的时间和空间纬度的综合信息来试图研究经济变化趋势的技术。政府为了提高
经济政策的有效性,就需要通过更加准确的预测来分析和判断信息,因此政府部门需要不断追求更加准确的经济预测技术,才能在复杂的经济形式变化中
掌握主动权。

除了经济发展的需要,为了规避国民经济的风险,把控我国宏观经济安全大局,也对政府更加准确地把握和预测国民经济提出了要求。
当前我国正处于经济和社会向前发展的关键时期,经济虽然多年持续高速发展,但是,经济结构发展不均衡的现象逐渐显现,
社会经济系统的复杂性也日渐增强。这就更需要对经济系统进行实时准确的检测和对未来经济形势准确预测,提前做好准备工作,
防范经济增长的重大风险。经济风险不仅仅和宏观经济运行状况紧密关联,同样和国计民生密不可分,
不管是通过对经济系统的宏观把控还是对经济运行中某些指标的局部观察,我们都可以了解经济运行的一些趋势,
并且通过对经济中的某些重要运行趋势进行预测,从而规避经济风险,稳定宏观经济安全大局。
把控宏观经济安全大局对于国民经济平稳快速发展、改善国计民生、促进经济又好又快发展、促进国家安全稳定有重要的战略和实际意义。
习近平总书记于2014年首次提出总体国家安全观,并构建了集政治安全、国土安全、军事安全、经济安全和文化安全等领域为一体的国家安全体系。
其中,经济安全是国家安全体系的重要组成部分,是国家安全的基础。宏观经济安全问题需研究1个或多个指标(如经济增长量)与其他指标(如资本、劳动力、人口和技术等)相互间的关系,
从而揭示经济现象背后的经济规律,发现经济运行风险点,制定经济政策。为了避免宏观经济中的冲击带来的风险,
需要对宏观经济总量进行实时预测和短期预测来减少不确定性,为宏观经济调控政策提供决策支持。通过对宏观经济指标进行有效监测和预测,
并提前识别和防范相关风险,对保证国民经济稳步增长具有重要意义。

宏观经济预测方法繁多,有的经济学家使用经济学方程组来给宏观经济系统建立模型,
从而通过模型来对某些宏观经济指标进行预测并且给出经济学分析。
然而国民经济是一个非常复杂的整体,它是一个内部极其复杂的系统,并且经常伴随难以预测的外部冲击,对于复杂的经济变量,想到做到准确预测往往非常艰难。
虽然现在传统的经济学理论和现代计算机技术的结合下,已经发展出了许多预测方法,并且学者们仍在不断利用新的理论和技术来完善这些已有的方法。然而,
在我国国情下,进行准确的经济预测仍然非常困难。现代经济系统极为复杂,
意味着使用这种方法我们就可能需要建立非常复杂的经济分析模型,有可能导致分析困难,并且往往预测结果和真实的经济运行结果有较大出入。
因为系统过于复杂,学者们就开始使用基于历史数据进行向前预测的经验方法,这种方法大量应用数据和数据分析、大数据技术和统计学知识,
并且融合了成熟的经济学理论和模型,往往能够起到不错的效果。在现代经济系统中,经济社会中每时每刻都在产生大量的数据,
这些海量的经济活动数据给经济预测提供了丰富的数据支撑,如何有效地利用大数据时代丰富的数据资源进行宏观经济现时预测已成为一个重要的研究课题。
大数据具有高维度、数量多、数据结构复杂和存储、计算困难等特征。但是结合计算机科学和统计学等方面的技术和理论知识,
我们能够采取一些方法来操纵数据,丛中过挖掘提炼出有效的信息。关于这个方面的研究正在火热进行,人们不断尝试使用更新的技术来处理数据,
提取信息,希望利用高维度下丰富多面的经济信息来对未来的经济发展趋势作较为准确的分析。

本文的研究意义在于,从主流的基于历史数据进行向前预测的方法中发现了问题,即使用近似因子模型进行宏观经济预测中存在稳健性方面的缺陷,通过
引入$L_1$稳健因子模型,给处理带有大量缺失值和离群值的宏观经济数据提供了更好的预测效果,在一定意义上提高了使用因子模型进行经济预测的准确性。

\subsection{国内外研究现状}
    \subsubsection{宏观经济预测}
    宏观经济预测
    \subsubsection{因子模型}
\subsection{研究内容及技术路线}
\subsection{研究创新}
\subsection{本文结构}