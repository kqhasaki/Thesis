

\centerline{\large\heiti 摘\ 要}

\vspace{2ex}
关注宏观经济的实时变化、对未来经济形势做出分析是一国政府制定经济政策的重要前提。
这需要时刻对宏观经济数据进行有效分析,而这些数据往往规模庞大、维数众多,难以采用传统手段进行建模。
因子分析是一种常用的数据降维手段,通过少数几个潜在的公共因子可以解释高维观测变量,
从而为建模和预测提供了方便。
因此因子分析成为一种分析高维宏观经济数据的有效方式,在理论和实践上都具有重要作用。
传统因子分析采用基于$L_2$范数的主成分分析法来估计因子载荷和因子得分,该方法应对
分布上具有重尾特征的宏观经济数据不够稳健。

本文采用$L_1$范数主成分分析来估计因子模型。首先阐述了一种求解$L_1$范数主成分分析的交替
凸优化算法,该算法可以求解$L_1$重构误差最小化问题,
随后通过模拟实验验证了$L_1$范数主成分分析对离群值数据的稳健性。
接下来进行实证研究,我们搜集了部分
国内月度宏观经济数据,观察到众多经济变量的分布呈现重尾特征。在该数据集上,分别采用
$L_1$范数主成分分析和$L_2$范数主成分分析进行估计得到静态因子,随后将静态因子应用于扩散指数
模型进行经济预测。实证结果显示,$L_1$范数主成分分析同样适用于静态因子的估计,并且更具稳健性与准确性。

交替凸优化算法是求解$L_1$范数主成分分析的经典算法,然而它需要在迭代中通过线性规划方法求解多个最小绝对值回归问题,
计算性能较差。本文对最小绝对值回归的性能问题进行了研究,介绍了两种较新的求解算法:聚类——迭代拆解算法和
基于替代变量的牛顿迭代方法。通过模拟实验,对两种算法的
性能表现做出了比较分析,并对二者的应用提出了一些建议。
% 另外,本文尝试结合两种算法的设计思想,提出了一种
% 结合二者优点的优化算法,并进行了模拟实验论证了该算法在计算效率上具有一定优势。

结合最小绝对值回归的优化算法,本文对交替凸优化算法做出了改进。通过模拟实验,我们发现改进后的交替凸优化算法
在求解大规模因子分析问题时有明显的效率提升。另外,还可以从投影$L_1$范数最大化角度来进行$L_1$主成分分析,
本文介绍了一种求解该问题的贪婪算法,并在国内月度宏观经济数据上进行了实证研究。实证结果显示,
采用两种$L_1$范数主成分分析法可得到不同的因子估计,都具有良好的预测表现,并且具有稳健性和准确性。


\vspace{2ex}
{\heiti 关键词}:$L_1$范数主成分分析;因子分析;最小绝对值回归;扩散指数模型
\thispagestyle{plain}

\newpage