\centerline{\large\bf Abstract}
\vspace{2ex}
Monitoring real-time changes in the macro economy and analyzing 
the future economic situation are substaintial prerequisites for 
a country's government to formulate economic policies.
Therefore it is necessary that effective analysis of macroeconomic data can be done. 
However the macroeconomic data is usually large in scale
and has many dimensions, making it difficult to model with traditional methods.
Factor analysis is a commonly used method for data dimensionality reduction, through which
one can explain high-dimensional observation variables by several potential common factors.
This provides convenience for modeling and forecasting.
Therefore, factor analysis has become an effective way to analyze 
high-dimensional macroeconomic data. And it has been playing a more and more
 important role in both theory and practice.
 Traditional factor analysis uses principal component analysis based 
 on $L_2$ norm to estimate factor loadings and factor scores. Hence it is not robust enough
  to deal with
macroeconomic data with heavy tail characteristics in distribution.

\thispagestyle{plain}
In this paper we use principal component analysis based on $L_1$ norm to 
estimate factor model. First, an alternating convex programming method , which solves 
the problem of minimizing $L_1$ norm reconstruction error 
is introduced. Then an simulation experiment is conducted to verify the robustness of $L_1$ norm
principal component analysis. 
Next for empirical research, we collected a set of
domestic macroeconomic data in which the economic variables are observed at a monthly basis.
It is observed that the distribution of many economic variables presents a heavy-tailed characteristic 
in distribution.
On this data set, both
$L_1$ norm principal component analysis and $L_2$ 
norm principal component analysis are used to estimate the static factors. Then 
those series of static factors are used for macroeconomic predicting through the diffusion index model.
The empirical results show that principal component analysis based on
$L_1$ norm is also applicable to the estimation of static factors, 
while being more robust and accurate.

While the alternating convex optimization algorithm is a classic 
algorithm for solving the principal component analysis based on $L_1$ norm, it has 
poor calculation performance because it
needs to solve multiple least absolute value regression 
problems through linear programming method at each iteration.
This paper studies the performance problem of least absolute 
value regression and introduces two newest algorithms: 
clustering-and-iterative-disassembling algorithm and
one Newton method based on substitution variable. 
Through simulation experiments, the two algorithms' performances
are compared and analyzed,and some suggestions are made for their application.
% In addition, this article attempts to propose a
% an optimized algorithm that combines the advantages of those two.
% A simulation experiment is conducted  demonstrating that the new algorithm 
% has certain advantages in computational efficiency.

\thispagestyle{plain}

Based on the newest algorithms for least absolute value regression, 
we can give some improvement to the alternating convex programming method.
Through simulation experiments, we found that the improved alternating convex optimization algorithm
shows a significant improvement in efficiency when solving large-scale factor analysis problems. 
In addition, principal component analysis of $L_1$ can also be performed 
from the perspective of maximizing $L_1$ norm of the low-demension projection.
We introduce a greedy algorithm to solve this problem. Then an empirical research is
conducted showing that both two kinds of principal component analysis methods
based on $L_1$ norm can estimate factor models well, 
at the mean time being more robust and accurate.

\vspace{2ex}
{\bf keywords:} \ $L_1$ norm principal component analysis, factor analysis,
least absolute value regression, diffusion index model

\thispagestyle{plain}
\newpage